\documentclass[a4paper,english]{lipics-v2019}

% overfull boxes debug
\overfullrule=1mm

\nolinenumbers

\usepackage{bussproofs}
\EnableBpAbbreviations

\newcommand{\agdaSymb}[1]{\mathsf{#1}}
\newcommand{\agdaKW}[1]{\mathbf{#1}}

\newcommand{\ind}{\hspace{1em}}

\newcommand{\data}{\agdaKW{data}}
\newcommand{\where}{\agdaKW{where}}
\newcommand{\Alet}{\agdaKW{let}}
\newcommand{\Ain}{\agdaKW{in}}

\newcommand{\Set}{\agdaSymb{Set}}
\newcommand{\Prop}{\agdaSymb{Prop}}
\newcommand{\Ty}{\agdaSymb{Ty}}
\newcommand{\Con}{\agdaSymb{Con}}
\newcommand{\Tms}{\agdaSymb{Tms}}
\newcommand{\Tm}{\agdaSymb{Tm}}
\newcommand{\id}{\agdaSymb{id}}
\newcommand{\app}{\agdaSymb{app}}
\newcommand{\lam}{\lambda}
\newcommand{\vz}{\agdaSymb{vz}}
\newcommand{\vs}{\agdaSymb{vs}}
\newcommand{\Var}{\agdaSymb{Var}}
\newcommand{\Vars}{\agdaSymb{Vars}}
\newcommand{\wk}{\agdaSymb{wk}}
%\newcommand{\Ne}{\agdaSymb{Ne}}
\newcommand{\Val}{\agdaSymb{Val}}
\newcommand{\Env}{\agdaSymb{Env}}
\newcommand{\NV}{\agdaSymb{NV}}
\newcommand{\var}{\agdaSymb{var}}
\newcommand{\neu}{\agdaSymb{neu}}
\newcommand{\clos}{\agdaSymb{clos}}
\newcommand{\qVal}{\agdaSymb{qVal}}
\newcommand{\qEnv}{\agdaSymb{qEnv}}
\newcommand{\idenv}{\agdaSymb{idenv}}
\newcommand{\Nf}{\agdaSymb{Nf}}
\newcommand{\NN}{\agdaSymb{NN}}
\newcommand{\eval}{\agdaSymb{eval}}
\newcommand{\evals}{\agdaSymb{evals}}
\newcommand{\q}{\agdaSymb{quote}}
\newcommand{\qn}{\agdaSymb{quoten}}
\newcommand{\norm}{\agdaSymb{norm}}
\newcommand{\scv}{\agdaSymb{scv}}
\newcommand{\sce}{\agdaSymb{sce}}
\newcommand{\fst}{\agdaSymb{fst}}
\newcommand{\snd}{\agdaSymb{snd}}
\newcommand{\U}{\agdaSymb{U}}
\newcommand{\El}{\agdaSymb{El}}
\newcommand{\TV}{\agdaSymb{TV}}
\newcommand{\Sk}{\agdaSymb{Sk}}
\newcommand{\base}{\agdaSymb{base}}
\newcommand{\skel}{\agdaSymb{skeleton}}
\newcommand{\isSet}{\agdaSymb{set}}

\newcommand{\cul}{\ulcorner}
\newcommand{\cur}{\urcorner}
\newcommand{\Ra}{\Rightarrow}
\newcommand{\Da}{\Downarrow}
\newcommand{\Beq}{\simeq_{\beta\eta}}


\bibliographystyle{plainurl} % Mandatory style

\title{Big Step Normalisation for Type Theory}

\author{Thorsten Altenkirch}
{School for Computer Science, University of Nottingham, United Kingdom}
{txa@cs.nott.ac.uk}{}{}

\author{Colin Geniet}
{Computer Science Department, ENS Paris-Saclay, France}
{colin.geniet@ens-paris-saclay.fr}{}{}

\authorrunning{T. Altenkirch and C. Geniet}

\Copyright{Thorsten Altenkirch and Colin Geniet}

% mandatory: Please choose ACM 2012 classifications from https://dl.acm.org/ccs/ccs_flat.cfm 
\ccsdesc[500]{Theory of computation~Type theory}

\keywords{Normalisation, type theory}

\supplement{https://github.com/colingeniet/big-step-normalisation}

%Editor-only macros:: begin (do not touch as author)%%%%%%%%%%%%%%%%%%%%%%%%%%%%%%%%%%
\EventEditors{}
\EventNoEds{0}
\EventLongTitle{}
\EventShortTitle{}
\EventAcronym{}
\EventYear{}
\EventDate{}
\EventLocation{}
\EventLogo{}
\SeriesVolume{}
\ArticleNo{}
%%%%%%%%%%%%%%%%%%%%%%%%%%%%%%%%%%%%%%%%%%%%%%%%%%%%%%

\begin{document}
\maketitle

\begin{abstract}
  Big step normalisation is a normalisation method for typed lambda-calculi
  which relies on a purely syntactic, inductively defined evaluator. We
  generalise big step normalisation to a minimalist dependent type theory.
  Compared to previous presentations of big step normalisation for e.g.\
  the simply typed lambda-calculus, we use a quotiented syntax, which reduces
  the syntactic complexity introduced by dependent types. Most of the proof has
  been formalised using Agda.
\end{abstract}

\section{Introduction}
\subsection{Normalisation}
In the context of typed lambda-calculi, normalisation refers to the process of
computing a canonical representative, called normal form, in each
$\beta\eta$-equivalence class of terms.

More formally, a normalisation function can be defined as a set of normal forms,
together with two (computable) maps, $\norm$ from terms to normal forms, and the
embedding $\cul\_\cur$ of normal forms into terms, satisfying
\begin{description}
\item[soundness] If $u \Beq v$, then $\norm\ u \equiv \norm\ v$
\item[completeness] For every term $u$, $\cul \norm\ u \cur \Beq u$
\item[stability] For every normal form $n$, $\norm\ \cul n \cur \equiv n$
\end{description}

The traditional way to define a normalisation function is through rewriting
theory. One proves that $\beta\eta$-reduction is confluent, and strongly normalising
on typed terms. Normal forms are defined as terms which can not be
$\beta\eta$-reduced, and normalisation is done by repetitively reducing a term
until a normal form is reached. Strong normalisation and confluence ensure the
correctness of the definition. Soundness also follows from confluence, while
completeness and stability are immediate. See for instance~\cite{girard1989proofs}
for a detailed proof of this result for the simply-typed lambda-calculus and some
variants (System F, System T). Unfortunately, this method fails for some other
variants of the lambda-calculus. For instance, the lambda-calculus with explicit
substitutions is not strongly normalising~\cite{mellies1995explicit}, and the
lambda-calculus with coproduct types (i.e.\ disjoint union) is not
confluent~\cite{dougherty1995coproducts}.

A more modern approach is normalisation by evaluation~(NBE), introduced
by Berger and Schwichtenberg~\cite{berger1991inverse} for the simply-typed
lambda-calculus. The idea is to evaluate terms into a semantic models,
meaning for instance that $\lambda$-abstractions (syntactic functions) are interpreted
by true (semantic) functions. A map from the model into normal forms is then
defined, giving rise to the normalisation function by composition with evaluation.
This method was for instance used to prove decidability of equivalence for the
lambda-calculus with coproducts~\cite{altenkirch2001normalization}.

\subsection{Big Step Normalisation}
Big step normalisation~(BSN) is an alternative, purely syntactic
normalisation method, proposed in~\cite{chapman2009bsn} by Chapman and the
first author for the simply-typed lambda-calculus. It can be summarized as
follows.

The normalisation function is defined in two parts. First, terms are
evaluated by an environment machine, yielding syntactic values. Then, values are
mapped to normal forms by a function named $\q$. Normalisation of terms is done
by evaluating in the identity environment, then applying $\q$ on the resulting value.

Evaluation and normalisation both have fairly simple definitions, but are
not structurally recursive, meaning that their termination is not obvious.
To prove termination, a Tait-style predicate~\cite{tait1967} called
\emph{strong computability} is defined on values:
\begin{itemize}
\item A value $v$ of the base type is strongly computable if normalisation
  terminates on $v$.
\item A value $f$ of type $A \Ra B$ is strongly computable if for any $v$ strongly
  computable of type $A$, the application of $f$ to $v$ terminates and the
  result is strongly computable.
\end{itemize}
The following results can then be proved.
\begin{itemize}
\item $\q$ terminates on any strongly computable value, and conversly any
  neutral value for which $\q$ terminates is strongly computable.
\item Evaluation of a term in a strongly computable environment terminates
  and the result is strongly computable.
\end{itemize}
Termination of normalisation follows from those results. Completeness and
stability are straight forward. The proof of soundness relies on a binary
relation on values, somewhat similar to strong computability.

\subsection{BSN for Type Theory}
Chapman also considered BSN for dependent type theory in~\cite{chapman2009type},
but did not provide a full proof of correctness, owing to the important
syntactic complexity added by the introduction of dependent types.

In this work, we propose some methods to simplify the proof of BSN in the case
of dependent types, allowing us to complete it.
The most important change is the use of the quotiented syntax of type theory
proposed in~\cite{kaposi2016type}. By only considering terms quotiented by
$\beta\eta$-equivalence, the syntax becomes significantly lighter. For instance,
the coercion constructors which form a large part of the syntactic boilerplate
encountered in~\cite{chapman2009type} become unnecessary.

In the context of a quotiented syntax, the notion of normalisation changes
slightly. Let us replace equivalence of terms by equality of quotiented terms
in our first definition of a normalisation function. Then soundness simply
states that $\norm$ is correctly defined on the quotiented syntax, while
completeness and stability state that $\norm$ and $\cul\_\cur$ are inverse of
each other.

This leads to the following definition proposed in~\cite{kaposi2016normalisation},
which we use in this work: a normalisation function is simply an isomorphism
between quotiented terms and normal forms. Obviously, this definition requires
a sensible notion of normal forms --- the identity should not be considered as
a normalisation function. Generally, normal forms are expected to have a simple
inductive definition, which ensures decidability of equality.\footnote{
  In the unquotiented case, the embedding of normal forms into terms (which can
  be proved to be injective) ensures that equality of normal forms is decidable,
  hence why no such restriction was required.
}

\subsection{Structure of the Paper}
Section~\ref{sec:theory} summarises the metatheory in which this paper is
presented, and some of the notations and conventions we use.

Section~\ref{sec:syntax} presents the quotiented syntax of type theory.

Section~\ref{sec:weakening} introduces a notion of weakening of contexts.

Section~\ref{sec:normalisation} defines big step normalisation itself. Because
it is not a priori clear that BSN defines a correct function (termination for
instance is problematic) we formally define normalisation by its big step
semantics, i.e.\ as a relation between inputs and outputs.

Section~\ref{sec:correctness} focuses on the two major correctness proofs:
termination and soundness. The proof of termination remains similar to the case
of simple types. The main difference is that we develop a simplified and
generalised induction principle for types, which allows us to manipulate
dependent types in almost the same way as simple types during the proof.
The proof of soundness for an unquotiented syntax seems much harder to
adapt, we instead provide a simple proof using soundness of NBE.

Finally, section~\ref{sec:cubical} explains how the proof of BSN can be
adapted to a cubical metatheory, using higher inductive types to encode
quotient inductive types.

\subsection{Related Work}
Big step normalisation was developed by Chapman and the first author for a
combinatory calculus~\cite{chapman2006tait}, and for the simply-typed
lambda-calculus~\cite{chapman2009bsn}. A generalisation to type theory was
also proposed~\cite{chapman2009type}, but without proof of correctness.
The present paper can be seen as a continuation of these works.

It is also interesting to consider the relation with the more recent work by
Kaposi and the first author which provides a concise, quotiented syntax of
type theory within (a larger) type theory~\cite{kaposi2016type}, and formalises
normalisation by evaluation in this syntax~\cite{kaposi2016normalisation}.

\section{Metatheory and Notations}
\label{sec:theory}
The present work has been formalised using a cubical metatheory~\cite{cchm}
implemented by Agda~\cite{norell2007agda}. This cubical theory provides a simple
way to define quotient inductive inductive types (QIIT, cf.~\cite{kaposi2016type})
as a special case of higher inductive types. However, for simplicity, this paper
is presented in a strict, intentional Martin-Löf Type Theory, extended with
QIIT. See section~\ref{sec:cubical} for for more details on the implementation
of the proof in a cubical metatheory.

Our metatheoretic notations are loosely based on the syntax of Agda. Function
types are written as $(x : A) \to B$, or simply $A \to B$ for non-dependent
functions. For ease of notations, we sometimes use infix arguments, denoted by
underscores, e.g.\ $\_,\_$ applied to $x$ and $y$ is written as $x,y$.
Functions with implicit arguments are defined as $f : \{x : A\} \to B$, and the
argument can be either omitted, or given in subscript as $f_x$. Sum types
(dependent pairs) are denoted by $\Sigma(x : A),\ B$. Omitted arguments are
implicitly universally quantified, e.g.\ $(y : A\ x) \to B$ is understood as
$\forall\,x, (y : A\ x) \to B$. The universe of types is denoted by $\Set$.
Inductive types are introduced by $\data$, followed by the sort of the defined
type, and the types the constructors. Inductive functions are defined by
pattern-matching. For instance:
\begin{flalign*}
  & \data\ \mathbb{N} : \Set\ \where
  && \_+\_ : \mathbb{N} \to \mathbb{N} \to \mathbb{N} && \\ & \ind
  \begin{alignedat}{2}
    & 0 : && \mathbb{N} \\
    & S : && \mathbb{N} \to \mathbb{N}
  \end{alignedat} &&
  \begin{alignedat}{2}
    & n + 0 && = n \\
    & n + (S\ m) && = S\ (n + m)
  \end{alignedat}
\end{flalign*}
Finally, the equality type is denoted by $x \equiv y$, while $=$ is only used in
definitions. The transport of $x : P\ a$ along an equality $p : a \equiv b$ is
denoted by $_{p*}x : P\ b$. If $p : a \equiv b$, the type of dependent equalities
between $x : P\ a$ and $y : P\ b$ lying over $p$ is denoted by $x \equiv^p y$.
For simplicity and readability, transports and dependent equality types will
be omitted starting from section~\ref{sec:weakening}.

\section{Quotiented Syntax of Type Theory}
\label{sec:syntax}
This section introduces the QIIT syntax of type theory used for this proof, as
proposed by Kaposi and the first author in~\cite{kaposi2016type}. It is an
intrinsically typed syntax, with De Bruijn indices, and explicit substitutions.

Contexts, types, substitutions and terms are mutually defined.
We denote contexts by $\Gamma,\Delta,\Theta,\Phi$, types by $A,B,C$,
substitutions by $\sigma,\nu,\delta$, and terms by $s,t,u$.
\begin{align*}
  & \data\ \Con : \Set \\
  & \data\ \Ty : \Con \to \Set \\
  & \data\ \Tms : \Con \to \Con \to \Set \\
  & \data\ \Tm : (\Gamma : \Con) \to \Ty\ \Gamma \to \Set
\end{align*}

Syntax constructors follow closely the definition of a category with
families~\cite{dybjer1995cwf} with product types. Contexts and substitutions
form a category, types are a presheaf, and terms are a family of presheaves over
types. There are three additional types constructors: the abstract base type
$\U$, the abstract base dependent family $\El$,\footnotemark{} and the  function
types $\Pi$. Each of these type constructors comes with an equation describing
its interaction with substitutions.

\footnotetext{%
  The names $\U$ and $\El$ reflect that $\U$ is an universe, in a sense of a
  type whose elements are themselves types (through $\El$). However, it is an
  abstract universe, in which no element can be built in an empty context.
}

The constructors are listed below, with regular constructors on the left, and
equality constructors on the right.
\begin{flalign*}
  % Contexts
  & \data\ \Con\ \where \\ & \ind
  \begin{alignedat}{2}
    & \bullet && : \Con \\
    & \_,\_ && : (\Gamma : \Con) \to \Ty\ \Gamma \to \Con
  \end{alignedat} \\
  % Types
  & \data\ \Ty\ \where && \data\ \Ty\ \where \\ & \ind
  \begin{alignedat}{2}
    & \_[\_] && : \Ty\ \Delta \to \Tms\ \Gamma\ \Delta \to \Ty\ \Gamma \\
    & \U && : \Ty\ \Gamma \\
    & \El && : \Tm\ \Gamma\ \U \to \Ty\ \Gamma \\
    & \Pi && : (A : \Ty\ \Gamma) \to \Ty\ (\Gamma,A) \to \Ty\ \Gamma \\ &
  \end{alignedat} &&
  \begin{alignedat}{2}
    & [\id] && : A [ \id ] \equiv A \\
    & [\circ] && : A [ \sigma \circ \nu ] \equiv A [ \sigma ] [ \nu ] \\
    & \U[] && : \U [ \sigma ] \equiv \U \\
    & \El[] && : (\El\ u) [ \sigma ] \equiv \El(_{\U[] *} u [ \sigma ]) \\
    & \Pi[] && : (\Pi\ A\ B) [ \sigma ] \equiv \Pi\ A[\sigma]\ B[\sigma \uparrow A]
  \end{alignedat}
  \displaybreak[0] \\
  % Substitutions
  & \data\ \Tms\ \where && \data\ \Tms\ \where \\ & \ind
  \begin{alignedat}{3}
    & \id       && :\ && \Tms\ \Gamma\ \Gamma \\
    & \_\circ\_ && : && \Tms\ \Delta\ \Theta \to \Tms\ \Gamma\ \Delta\ \to \\
    &           &&   && \Tms\ \Gamma\ \Theta \\
    & \epsilon  && : && \Tms\ \Gamma\ \bullet \\
    & \_,\_     && : && (\sigma : \Tms\ \Gamma\ \Delta) \to \Tm\ \Gamma\ A[\sigma] \to \\
    &           &&   && \Tms\ \Gamma\ (\Delta,A) \\
    & \pi_1     && : && \Tms\ \Gamma\ (\Delta,A) \to \Tms\ \Gamma\ \Delta
  \end{alignedat} &&
  \begin{alignedat}{2}
    & \id\circ && : \id \circ \sigma \equiv \sigma \\
    & \circ\id && : \sigma \circ \id \equiv \sigma \\
    & \circ\circ && : (\sigma \circ \nu) \circ \delta \equiv \sigma \circ (\nu \circ \delta) \\
    & \epsilon\eta && : \{\sigma : \Tms\ \Gamma\ \bullet\} \to \sigma \equiv \epsilon \\
    & \pi_1\beta && : \pi_1(\sigma,u) \equiv \sigma \\
    & \pi\eta  && : \pi_1\,\sigma , \pi_2\,\sigma \equiv \sigma \\
    & ,\circ  && : (\sigma,u) \circ \nu \equiv (\sigma \circ \nu),(_{[\circ]^{-1} *}u[\nu])
  \end{alignedat}
  \displaybreak[0] \\
  % Terms
  & \data\ \Tm\ \where && \data\ \Tm\ \where \\ & \ind
  \begin{alignedat}{3}
    & \pi_2     && :\ && (\sigma : \Tms\ \Gamma\ (\Delta,A)) \to \Tm\ \Gamma\ A[\pi_1 \sigma] \\
    & \_[\_]    && : && \Tm\ \Delta\ A \to (\sigma : \Tms\ \Gamma\ \Delta) \to \\
    &           &&   && \Tm\ \Gamma\ A[\sigma] \\
    & \lam      && : && \Tm\ (\Gamma,A)\ B \to \Tm\ \Gamma\ (\Pi\ A\ B) \\
    & \app      && : && \Tm\ \Gamma\ (\Pi\ A\ B) \to \Tm\ (\Gamma,A)\ B
  \end{alignedat} &&
  \begin{alignedat}{2}
    & \pi_2\beta && : \pi_2(\sigma,u) \equiv^{\pi_1\beta} u \\
    & \beta    && : \app\ (\lam u) \equiv u \\
    & \eta     && : \lam (\app\, u) \equiv u \\
    & \lam[]   && : (\lam u)[\sigma] \equiv^{\Pi[]} \lam(u[\sigma\uparrow A]) \\ &
  \end{alignedat}
\end{flalign*}
Equations $\Pi[]$ and $\lambda[]$ use the lifting of a substitution by a type,
defined as follows.
\begin{align*}
  & \_\uparrow\_      : (\sigma : \Tms\ \Gamma\ \Delta) \to (A:\Ty) \to
                        \Tms\ (\Gamma,A[\sigma])\ (\Delta,A) \\
  & \sigma \uparrow A = (\sigma \circ \pi_1\,\id) , (_{[\circ]*}\pi_2\,\id)
\end{align*}

While this syntax does not contain the usual application of the lambda-calculus,
it can reconstructed by combining the categorical application and with the
substitution of a single term $<\_>$.
\begin{alignat*}{2}
  & <\_> && : \Tm\ \Gamma\ A \to \Tms\ \Gamma\ (\Gamma,A) \\
  & <u> && = \id\,,\,_{[\id]^{-1} *}u \\
  & \_\$\_ && : \Tm\ \Gamma\ (\Pi\ A\ B) \to (u : \Tm\ \Gamma\ A) \to \Tm\ \Gamma\ B[<u>] \\
  & f\ \$\ u && = (\app\, f)[<u>]
\end{alignat*}


\section{Weakenings}
\label{sec:weakening}
The proof of BSN requires a notion of weakening of a context. For this purpose,
we use renamings, i.e.\ lists of variables. This section defines variables and
renamings, as presented in~\cite{kaposi2016normalisation}.

Variables, denoted by $x,y,z$ are defined as typed De Bruijn indices. There is
an embedding of variables into terms.
\begin{flalign*}
  & \data\ \Var\ : (\Gamma : \Con) \to \Ty\ \Gamma \to \Set\ \where
  && \cul\_\cur : \Var\ \Gamma\ A \to \Tm\ \Gamma\ A \\ & \ind
  \begin{alignedat}{2}
    & \vz && : \Var\ (\Gamma,A)\ (A[\pi_1\,\id]) \\
    & \vs && : \Var\ \Gamma\ A \to \Var\ (\Gamma,B)\ (A[\pi_1\,\id])
  \end{alignedat} &&
  \begin{alignedat}{2}
    & \cul \vz \cur && = \pi_2\,\id \\
    & \cul \vs\,x \cur && = \cul x \cur [\pi_1\,\id]
  \end{alignedat}
\end{flalign*}
Renamings, denoted by $\alpha,\beta,\gamma$, and their embedding into
substitutions are mutually defined.
\begin{flalign*}
  & \data\ \Vars\ : \Con \to \Con \to \Set\ \where
  && \cul\_\cur : \Vars\ \Gamma\ \Delta \to \Tms\ \Gamma\ \Delta \\ & \ind
  \begin{alignedat}{2}
    & \epsilon && : \Vars\ \Gamma\ \epsilon \\
    & \_,\_ && : (\alpha : \Vars\ \Gamma\ \Delta) \to \Var\ \Gamma\ A[\cul \alpha \cur]
    \to \Var\ \Gamma\ (\Delta,A)
  \end{alignedat} &&
  \begin{alignedat}{2}
    & \cul \epsilon \cur && = \epsilon \\
    & \cul \alpha,x \cur && = \cul \alpha \cur , \cul x \cur
  \end{alignedat}
\end{flalign*}

Definitions of identity and composition require the appropriate auxiliary
functions. The identity uses the weakening $\wk$ of a renaming by a type,
and composition uses the application $\_[\_]$ of a renaming to a variable.
These functions all commute with embeddings of variables and renamings.
We omit the definitions, which are relatively simple.
\begin{flalign*} &
  \begin{alignedat}{2}
    & \wk && : (A : \Ty\ \Gamma) \to \Vars\ \Gamma\ \Delta \to \Vars\ (\Gamma,A)\ \Delta \\
    & \id && : \{\Gamma : \Con\} \to \Vars\ \Gamma\ \Gamma \\
    & \_[\_] && : \Var\ \Delta\ A \to (\alpha : \Vars\ \Gamma\ \Delta) \to \Var\ \Gamma\ A[\cul\alpha\cur] \\
    & \_\circ\_ && : \Vars\ \Delta\ \Theta \to \Vars\ \Gamma\ \Delta \to \Vars\ \Gamma\ \Theta
  \end{alignedat} &&
  \begin{alignedat}{2}
    & \cul\wk\cur && : \cul \wk\,\alpha \cur \equiv \cul \alpha \cur [ \pi_1\,\id ] \\
    & \cul\id\cur && : \cul \id \cur \equiv \id \\
    & \cul[]\cur && : \cul x [ \alpha ] \cur \equiv \cul x \cur [ \cul \alpha \cur ] \\
    & \cul\circ\cur && : \cul \alpha \circ \beta \cur \equiv \cul \alpha \cur \circ \cul \beta \cur
  \end{alignedat}
\end{flalign*}
With these operations, it can be verified that contexts and renamings form a
category.

Types, terms, and substitutions can be weakened by renamings by applying the
renaming seen as a substitution through its embedding. It is easy to verify that
weakening respects identity and composition of renamings. Precisely, types and
substitutions are presheaves on the category of renamings, while terms are a
family of presheaves over types.
\begin{flalign*} &
  \begin{alignedat}{2}
    & \_\ ^{+\_} && : \Ty\ \Delta \to \Vars\ \Gamma\ \Delta \to \Ty\ \Gamma \\
    & A^{+\alpha} && = A [\cul\alpha\cur] \\
    & \_\ ^{+\_} && : \Tm\ \Delta\ A \to (\alpha : \Vars\ \Gamma\ \Delta) \to \Tm\ \Gamma\ A^{+\alpha} \\
    & u^{+\alpha} && = u [\cul\alpha\cur] \\
    & \_\ ^{+\_} && : \Tms\ \Delta\ \Theta \to \Vars\ \Gamma\ \Delta \to \Tms\ \Gamma\ \Theta \\
    & \sigma^{+\alpha} && = \sigma \circ \cul\alpha\cur
  \end{alignedat} &&
  \begin{alignedat}{2}
  & +\id && : A^{+\id} \equiv A \\
  & +\circ && : A^{+ (\alpha \circ \beta)} \equiv (A^{+\alpha})^{+\beta} \\
  & +\id && : u^{+\id} \equiv u \\
  & +\circ && : u^{+ (\alpha \circ \beta)} \equiv (u^{+\alpha})^{+\beta} \\
  & +\id && : \sigma^{+\id} \equiv \sigma \\
  & +\circ && : \sigma^{+ (\alpha \circ \beta)} \equiv (\sigma^{+\alpha})^{+\beta}
  \end{alignedat}
\end{flalign*}

This will be a general pattern in later constructions and proofs: families of
sets that will be defined (e.g.\ values, normal forms) have a presheaf-like
structure, which simply means that the elements can be weakened coherently.
Similarly, functions are natural transformations, i.e.\ are compatible
with weakening, and predicates are sub-presheaves, meaning that they are stable
under weakening. The corresponding proofs are typically straight forward, and
we will often not mention them. We abusively denote all weakenings by $\_\ ^{+\_}$.

Finally, given a type $A$, one may consider $\wk\ A\ \id : \Vars\ (\Gamma,A)\ \Gamma$,
the weakening of a context $\Gamma$ by $A$. We abuse notations and write
$u^{+A}$ for $u^{+ (\wk\ A\ \id)}$.


\section{Normalisation Relation}
\label{sec:normalisation}
This section defines big step normalisation for the previous quotiented syntax
of type theory. We will first define values, and the evaluation from terms to
values, then normal forms, and the function $\q$ mapping values to normal
forms.

\subsection{Values}
A value is either a closure, corresponding to the delayed evaluation of a
lambda-abstraction, or a neutral value, that is the stuck application of a
variable to values. This suggests the following mutual definition of values
(denoted by $v,w$), neutral values (denoted by $n$), and environments (lists of
values, denoted by $\rho,\omega$) --- together with the associated embeddings.
\begin{flalign*}
  & \data\ \Val : (\Gamma : \Con) \to \Ty\ \Gamma \to \Set\ \where
  && \cul\_\cur : \Val\ \Gamma\ A \to \Tm\ \Gamma\ A \\ & \ind
  \begin{alignedat}{2}
    & \neu   && : \NV\ \Gamma\ A \to \Val\ \Gamma\ A \\
    & \clos  && : \Tm\ (\Delta,A)\ B \to (\rho : \Env\ \Gamma\ \Delta) \to \\
    &        && \quad \Val\ \Gamma\ (\Pi\ A\ B)[\cul \rho \cur]
  \end{alignedat} &&
  \begin{alignedat}{2}
    & \cul \neu\ n \cur && = \cul n \cur \\
    & \cul \clos\ u\ \rho \cur && = (\lambda u)[\cul \rho \cur] \\ &
  \end{alignedat}
  \displaybreak[0] \\
  & \data\ \NV : (\Gamma : \Con) \to \Ty\ \Gamma \to \Set\ \where
  && \cul\_\cur : \NV\ \Gamma\ A \to \Tm\ \Gamma\ A \\ & \ind
  \begin{alignedat}{2}
    & \var && : \Var\ \Gamma\ A \to \NV\ \Gamma\ A \\
    & \app && : \NV\ \Gamma\ (\Pi\ A\ B) \to (v : \Val\ \Gamma\ A) \to \\
    &      && \quad \NV\ \Gamma\ B[<\cul v \cur>]
  \end{alignedat} &&
  \begin{alignedat}{2}
    & \cul \var\ x \cur && = \cul x \cur \\
    & \cul \app\ n\ v\cur && = \cul n \cur\,\$\,\cul v \cur \\ &
  \end{alignedat}
  \displaybreak[0] \\
  & \data\ \Env : \Con \to \Con \to \Set\ \where
  && \cul\_\cur : \Env\ \Gamma\ \Delta \to \Tms\ \Gamma\ \Delta \\ & \ind
  \begin{alignedat}{2}
    & \epsilon && : \Env\ \Gamma\ \bullet \\
    & \_,\_ && : (\rho : \Env\ \Gamma\ \Delta) \to \Val\ \Gamma\ A[\cul\rho\cur] \to \Env\ \Gamma\ (\Delta,A)
  \end{alignedat} &&
  \begin{alignedat}{2}
    & \cul \epsilon \cur && = \epsilon \\
    & \cul \rho,v \cur && = \cul \rho \cur\,\cul v \cur
  \end{alignedat}
\end{flalign*}

This definition would be used with an unquotiented syntax such as in~\cite{chapman2009bsn},
but is inappropriate for the quotiented syntax because some values are
equivalent when seen as terms (formally, have equal embeddings), but are not
equal. For instance, if one considers a closure $\clos\ u\ \rho$ in which the
body $u$ never refers to the environment $\rho$, then modifying $\rho$ will
yield a distinct, yet equivalent value. Consequently, the evaluation function
may map equivalent terms to distinct values, hence can not be defined on the
quotiented syntax.

To avoid this, values are quotiented, ensuring that their embedding is injective.
\begin{flalign*}
  & \data\ \Val\ \where \\
  & \ind \qVal : (v\ w : \Val\ \Gamma\ A) \to \cul v \cur \equiv \cul w \cur \to v \equiv w
\end{flalign*}
The corresponding result for environments can be proved by induction on contexts.
\[ \qEnv : (\rho\ \omega : \Env\ \Gamma\ \Delta) \to
  \cul \rho \cur \equiv \cul \omega \cur \to \rho \equiv \omega \]

To complete this definition, weakening is defined by induction on values,
neutral values, and environments. We omit the definitions and the associated lemmas.

Finally, we introduce two useful constructions on environments.
Projections are simply defined by pattern matching.
\begin{flalign*}
  & \fst : \Env\ \Gamma\ (\Delta,A) \to \Env\ \Gamma\ \Delta
  && \snd : (\rho : \Env\ \Gamma\ (\Delta,A)) \to \Val\ \Gamma\ A[\cul \fst\ \rho \cur] \\
  & \fst\ (\rho,v) = \rho
  && \snd\ (\rho,v) = v
\end{flalign*}
The identity environment is defined by induction on the context, and uses
weakening of environments.
\begin{flalign*}
  & \idenv : \{\Gamma : \Con\} \to \Env\ \Gamma\ \Gamma \\ &
  \begin{alignedat}{2}
    & \idenv_{\bullet} && = \epsilon \\
    & \idenv_{\Gamma,A} && = {\idenv_{\Gamma}}^{+A} \,,\, \neu\,(\var\ z)
  \end{alignedat}
\end{flalign*}


\subsection{Evaluation}
The first stage of normalisation is an environment machine, which evaluate
terms in an environment, and returns values. It consists of three mutually
defined functions: $\eval$ and $\evals$ evaluate terms and substitutions
respectively in an environment, while $\_@\_$ computes the application of an
value to another.

The evaluator could be presented as follows.
\begin{flalign*}
  & \eval : \Tm\ \Delta\ A \to (\rho : \Env\ \Gamma\ \Delta) \to \Val\ \Gamma\ A[\cul \rho \cur] \\ &
  \begin{alignedat}{2}
    & \eval\ (\pi_2\, \sigma)\ \rho && = \snd\ (\evals\ \sigma\ \rho) \\
    & \eval\ (u[\sigma])\ \rho && = \eval\ u\ (\evals\ \sigma\ \rho) \\
    & \eval\ (\lam u)\ \rho && = \clos\ u\ \rho \\
    & \eval\ (\app\ u)\ (\rho,v) && = (\eval\ u\ \rho)\ @\ v
  \end{alignedat}
  \displaybreak[0] \\
  & \evals : \Tms\ \Delta\ \Theta \to \Env\ \Gamma\ \Delta \to \Env\ \Gamma\ \Theta \\ &
  \begin{alignedat}{2}
    & \evals\ \id\ \rho && = \rho \\
    & \evals\ (\sigma \circ \nu)\ \rho && = \evals\ \sigma\ (\evals\ \nu\ \rho) \\
    & \evals\ \epsilon\ \rho && = \epsilon \\
    & \evals\ (\sigma,u)\ \rho && = (\evals\ \sigma\ \rho) , (\eval\ u\ \rho) \\
    & \evals\ (\pi_1\, \sigma)\ \rho && = \fst\ (\evals\ \sigma\ \rho)
  \end{alignedat}
  \displaybreak[0] \\
  & \_@\_ : \Val\ \Gamma\ (\Pi\ A\ B) \to (v : \Val\ \Gamma\ A) \to \Val\ \Gamma\ B[<\cul v \cur>] \\ &
  \begin{alignedat}{2}
    & (\clos\ u\ \rho)\ @\ v && = \eval\ u\ (\rho,v) \\
    & (\neu\ n)\ @\ v && = \neu\ (\app\ n\ v)
  \end{alignedat}
\end{flalign*}
Most cases are very straight forward. Note how evaluation of a lambda does
nothing, and merely returns a closure, delaying the evaluation of the body.
The latter occurs in the first case of $\_@\_$, as the application of
a closure to a value is computed by evaluating the body of the closure in the
extended environment.

However, there are several problems with the previous presentation of the
evaluator. For a start, the functions are defined by recursion on terms and
substitutions, which are QIIT, but we did not bother to verify that the equality
constructors are respected. Perhaps more worryingly, the function is not
structurally recursive, hence it is not clear that the evaluator terminates.

Proving that the definition of the evaluator as a function is correct is
non-trivial, but this issue can be delayed by considering its big step
semantics, which means that we redefine the evaluator as a relation between
inputs and outputs. For instance, we denote by $\eval\ t\ \rho \Da v$ the
proposition `$t$ evaluates to $v$ in environment $\rho$'. As an example, we
define the relation corresponding to $\eval$, the other cases being similar.
\begin{flalign*}
  & \data\ \eval\_\,\_\Da\_ : \Tm\ \Delta\ A \to \Env\ \Gamma\ \Delta \to
                              \Val\ \Gamma\ B \to \Prop\ \where \\ & \ind
  \begin{alignedat}{2}
    & \eval\pi_2 && : \evals\ \sigma\ \rho \Da (\omega,v) \to \eval\ (\pi_2\ \sigma)\ \rho \Da v \\
    & \eval[]    && : \evals\ \sigma\ \rho \Da \omega \to \eval\ u\ \omega \Da v \to
                      \eval\ (u[\sigma])\ \rho \Da v \\
    & \eval\lam  && : \eval\ (\lam u)\ \rho \Da (\clos\ u\ \rho) \\
    & \eval\app  && : \eval\ f\ \rho \Da g \to g\ @\ v \Da w \to \eval\ (\app\ f)\ (\rho,v) \Da w
  \end{alignedat} \\
  & \data\ \evals\_\,\_\Da\_ : \Tms\ \Delta\ \Theta \to \Env\ \Gamma\ \Delta \to
                               \Env\ \Gamma\ \Theta \to \Prop \\
  & \data\ \_@\_\Da\_ : \Val\ \Gamma\ A \to \Val\ \Gamma\ B \to \Val\ \Gamma\ C \to \Prop
\end{flalign*}
The types of the above relations may seem surprisingly imprecise. For instance,
the type of $\eval$ does not give any information on the type of the return
value --- it is a value of some unknown type $B$ --- whereas we know that
it should have type $A[<\cul \rho \cur>]$ when evaluating in environment $\rho$.
Similarly, we do not even require the first argument of $@$ to be a function.
While stricter types would be theoretically possible, this choice of loose types
simplifies the formal proofs by reducing the need for transports.
This is similar to the way using heterogeneous equality instead of the more
restrictive dependent equality types can be simplify matters.

Furthermore, nothing is lost by using these loose types, because the following
lemma implies that the expected restrictions are indeed satisfied.
\begin{lemma}
  \label{lem:evalCompl}
  \[
    \AXC{$\eval\ u\ \rho \Da v$}
    \UIC{$\cul v \cur \equiv u[ \cul\rho\cur ]$}
    \DP \quad
    \AXC{$\evals\ \sigma\ \rho \Da \omega$}
    \UIC{$\cul\omega\cur \equiv \sigma \circ \cul\rho\cur$}
    \DP \quad
    \AXC{$f\ @\ v \Da w$}
    \UIC{$\cul f \cur\ \$\ \cul v \cur \equiv \cul w \cur$}
    \DP
  \]
\end{lemma}
\begin{proof}
  By induction on the evaluation relation.
\end{proof}

A soundness property follows.
\begin{lemma}
  \label{lem:evalSound}
  \[
    \AXC{$\eval\ u\ \rho \Da v$}
    \AXC{$\eval\ u\ \rho \Da w$}
    \BIC{$v \equiv w$}
    \DP \quad
    \AXC{$\evals\ \sigma\ \rho \Da \omega$}
    \AXC{$\evals\ \sigma\ \rho \Da \delta$}
    \BIC{$\omega \equiv \delta$}
    \DP
  \]
  \[
    \AXC{$f\ @\ u \Da v$}
    \AXC{$f\ @\ u \Da w$}
    \BIC{$v \equiv w$}
    \DP
  \]
\end{lemma}
\begin{proof}
  Using lemma~\ref{lem:evalCompl}, and that embeddings of values and
  environments are injective.
\end{proof}

\subsection{Normal Forms}
Having defined the evaluator, we continue with the function $\q$ which maps
values to normal forms. The classic notion of \emph{$\eta$-long $\beta$-normal forms}
is used, which interestingly is shared with normalisation by evaluation
(cf.~\cite{kaposi2016normalisation}).

In common with values, normal forms are defined mutually with neutral normal
forms, i.e.\ the application of a variable to normal forms. An important
difference is that neutral normal forms are normal forms only for the base types
$\U$ and $\El$. This restriction ensures that normal forms are sufficiently
$\eta$-expanded.
\begin{flalign*}
  & \data\ \Nf : (\Gamma : \Con) \to \Ty\ \Gamma \to \Set\ \where
  && \cul\_\cur : \Nf\ \Gamma\ A \to \Tm\ \Gamma\ A \\ & \ind
  \begin{alignedat}{2}
    & \lam && : \Nf\ (\Gamma,A)\ B \to \Nf\ \Gamma (\Pi\ A\ B) \\
    & \neu\U && : \NN\ \Gamma\ \U \to \Nf\ \Gamma\ \U \\
    & \neu\El && : \NN\ \Gamma\ (\El\ u) \to \Nf\ \Gamma\ (\El\ u)
  \end{alignedat} &&
  \begin{alignedat}{2}
    & \cul \lam n \cur && = \lam\, \cul n \cur \\
    & \cul \neu\U\ n \cur && = \cul n \cur \\
    & \cul \neu\El\ n \cur && = \cul n \cur
  \end{alignedat} \\
  & \data\ \NN : (\Gamma : \Con) \to \Ty\ \Gamma \to \Set\ \where
  && \cul\_\cur : \NN\ \Gamma\ A \to \Tm\ \Gamma\ A \\ & \ind
  \begin{alignedat}{2}
    & \var && : \Var\ \Gamma\ A \to \NN\ \Gamma\ A \\
    & \app && : \NN\ \Gamma\ (\Pi\ A\ B) \to (n : \NN\ \Gamma\ A) \to \\
    &      && \quad \NN\ \Gamma\ B[<\cul n \cur>]
  \end{alignedat} &&
  \begin{alignedat}{2}
    & \cul \var\ x \cur && = \cul x \cur \\
    & \cul \app\ m\ n \cur && = \cul m \cur\,\$\,\cul n \cur \\ &
  \end{alignedat}
\end{flalign*}
Note that normal forms are indexed by regular types, and not a notion of normal
types. Indeed, normalising types and terms simultaneously only seems to
complicate matters, and it is easier to first normalise a term without worrying
about its type, then recursively normalise the type. A disadvantage of this
choice is that equality of normal forms is not a priori decidable, because it
would require to test equality of types, and in turn equality of terms.
However, this issue can be solved after the normalisation function is defined
by proving decidability of equality for terms, normal forms, and types
simultaneously, as shown in~\cite{kaposi2016normalisation}.

\subsection{Quote}
The function $\q$ is defined by induction on the type of the value, together
with $\qn$ which maps neutral values to neutral normal forms by recursively
applying $\q$. Like the evaluator, we begin with an informal definition as
a function, which is then translated to a relation.
\begin{flalign*}
  & \q : \{A : \Ty\} \to \Val\ \Gamma\ A \to \Nf\ \Gamma\ A \\ &
  \begin{alignedat}{2}
    & \q_{\U}\ (\neu\ v) && = \neu\U\ (\qn\ v) \\
    & \q_{(\El\ u)}\ (\neu\ v) && = \neu\El\ (\qn\ v) \\
    & \q_{(\Pi\ A\ B)}\ f       && = \lam (\q\ (f^{+A}\ @\ \neu\ (\var\ \vz)))
  \end{alignedat} \\
  & \qn : \NV\ \Gamma\ A \to \NN\ \Gamma\ A \\ &
  \begin{alignedat}{2}
    & \qn\ (\var\ x) && = \var\ x \\
    & \qn\ (\app\ f\ v) && = \app\ (\qn\ f)\ (\q\ v)
  \end{alignedat}
\end{flalign*}
A value of a base type is necessarily neutral, hence it suffice to use
$\qn$ in that case. For function types, the definition of normal forms
requires the result to be an abstraction. This is done by $\eta$-expending the
value, and applying $\q$ to the body of the resulting abstraction.
The $\eta$-expansion is somewhat technical to define. First, the function is
weakened ($f^{+A}$) to allow the introduction of a new variable represented
by the De Bruijn index $\vz$. This variable is turned into a value by the
$\var$ and $\neu$ constructors, and the weakened function is applied
using $@$, giving the body of the $\eta$-expansion.

Beside the problems of termination and correctness with regards to quotient
constructors which are shared with the evaluator, one may note that $\q$ is
not defined on the $\_[\_]$ type constructor. We will later show that
it can be inferred from the remaining cases and equality constraints. For now
we again ignore all issues by considering the big step semantics of $\q$,
with the following signature.
\begin{align*}
  & \q : \Val\ \Gamma\ A \to \Nf\ \Gamma\ A \to \Prop \\
  & \qn : \NV\ \Gamma\ A \to \NN\ \Gamma\ A \to \Prop
\end{align*}

A coherence result in the style of lemma~\ref{lem:evalCompl} is proved by
induction on the relation.
\begin{lemma}
  \label{lem:quoteCompl}
  \[
    \AXC{$\q\ v \Da n$}
    \UIC{$\cul n \cur \equiv \cul v \cur$}
    \DP \quad
    \AXC{$\qn\ m \Da n$}
    \UIC{$\cul n \cur \equiv \cul m \cur$}
    \DP
  \]
\end{lemma}

\subsection{Normalisation}
Finally, terms are normalised by evaluating in the identity environment and
applying $\q$.
\[ \norm\ u \Da n = \Sigma(v : \Val\ \Gamma\ A)\ \eval\ u\ \idenv \Da v \ \land \ \q\ v \Da n \]

With this definition, stability and completeness of BSN can already be proved.
\begin{proposition}[Completeness]
  \label{prop:completeness}
  \[
    \AXC{$\norm\ u \Da n$}
    \UIC{$\cul n \cur \equiv u$}
    \DP
  \]
\end{proposition}
\begin{proof}
  Immediate by lemmas~\ref{lem:evalCompl} and~\ref{lem:quoteCompl}.
\end{proof}
\begin{proposition}[Stability]
  \label{prop:stability}
  \[
    \AXC{$n : \Nf\ \Gamma\ A$}
    \UIC{$\norm\ \cul n \cur \Da n$}
    \DP\quad
    \AXC{$n : \NN\ \Gamma\ A$}
    \UIC{$\Sigma(v : \NV\ \Gamma\ A)\ \eval\ \cul n \cur \Da (\neu\ v) \ \land \ \qn\ v \Da n$}
    \DP
  \]
\end{proposition}
\begin{proof}
  By simultaneous induction on normal forms and neutral normal forms.
\end{proof}


\section{Correctness of BSN}
\label{sec:correctness}
Two main results must be proved in order to establish the correctness of BSN as
previously defined.
Termination states that the normalisation relation is defined on every term.
\[ \forall (u : \Tm\ \Gamma\ A),\ \exists (n : \Nf\ \Gamma\ A),\ \norm\ u \Da n \]
Soundness states that normalisation can only give one result for each term.
\[
  \AXC{$\norm\ u \Da n$}
  \AXC{$\norm\ u \Da m$}
  \BIC{$n \equiv m$}
  \DP
\]
Termination and soundness together imply that the normalisation relation defines
a function from terms to normal forms, and the remaining coherence properties
(completeness and stability) have already been proved in the previous section.

In this section, we first provide a short proof of soundness using known results
on NBE.

Next, we define evaluation of types and the notion of skeleton of a type, which
address two issues with the induction principle for types caused by the
introduction of dependent types. 

The rest of the proof of termination then closely follows the original
one~\cite{chapman2009bsn}, using the strong computability predicate.

\subsection{Soundness, for free}
The original presentation of BSN for the simply-typed lambda-calculus proves
soundness using a logical relation, similar to the use of strong computability
for termination, presented later in this section. Unfortunately, this proof
seems hard to adapt to the quotiented syntax.

However there is an alternative proof, much shorter if not as interesting.
The key observation is that BSN uses the same notion of normal forms as
normalisation by evaluation (cf.~\cite{kaposi2016normalisation} for a formal
proof of NBE --- we use the very same syntax and normal forms). A direct
consequence of the existence of a normalisation is that there is exactly one
normal form in each equivalence class of terms, which in the quotiented syntax
means that the embedding of normal forms is injective.
\begin{theorem}
  \label{thm:nbe}
  \[ \AXC{$n,m : \Nf\ \Gamma\ A$}\AXC{$\cul n \cur \equiv \cul m \cur$}\BIC{$n \equiv m$}\DP \]
\end{theorem}
\begin{proof}
  By soundness and stability of normalisation by evaluation.
\end{proof}

\begin{proposition}[Soundness]
  \label{prop:soundness}
  \[ \AXC{$\norm\ u \Da n$}\AXC{$\norm\ u \Da m$}\BIC{$n \equiv m$}\DP \]
\end{proposition}
\begin{proof}
  Immediate by proposition~\ref{prop:completeness} and theorem~\ref{thm:nbe}.
\end{proof}

It can of course be argued that defining a normalisation function using another
normalisation function defeats the object. However it may be interesting to
consider BSN not so much as alternative normalisation function than as an
alternative definition for the function obtained through NBE. This proof of
soundness becomes more sensible from this point of view: as soon as we prove
that the functions defined by NBE and BSN coincide (for which completeness of
BSN is a key result), all correctness properties which are known to hold for NBE
--- in particular soundness --- transfer to BSN.

\subsection{Evaluation of Types}
An interesting issue was mentioned while defining $\q$: the natural definition
is by induction on types, but only considers the constructors $\U$, $\El$, and
$\Pi$, forgetting both $\_[\_]$ and the quotient constructors. In this
subsection, we show that this type of definition is in fact always correct, by
defining substitution-free types, and proving that they are isomorphic to
regular types.
\begin{flalign*}
  & \data\ \TV : \Con \to \Set\ \where && \cul\_\cur : \TV\ \Gamma \to \Ty\ \Gamma && \\ & \ind
  \begin{alignedat}{2}
    & \U && : \TV\ \Gamma \\
    & \El && : \Tm\ \Gamma\ \U \to \TV\ \Gamma \\
    & \Pi && : (A : \TV\ \Gamma) \to \TV\ (\Gamma,\cul A \cur) \to \TV\ \Gamma
  \end{alignedat} &&
  \begin{alignedat}{2}
    & \cul \U \cur && = \U \\
    & \cul \El\ u \cur && = \El\ u \\
    & \cul \Pi\ A\ B \cur && = \Pi\ \cul A \cur\ \cul B \cur
  \end{alignedat}
\end{flalign*}

We will now define an evaluation function from types to substitution-free types,
which will be the inverse of the embedding $\cul\_\cur$. This requires to
interpret every remaining type constructors in substitution-free types.

First, the application of a substitution to a substitution-free type is defined
inductively.
\begin{flalign*}
  & \_[\_] : \TV\ \Delta \to \Tms\ \Gamma\ \Delta \to \TV\ \Gamma && \\ &
  \begin{alignedat}{2}
    & \U [\sigma] && = \U \\
    & (\El\ u) [\sigma] && = \El (u[\sigma]) \\
    & (\Pi\ A\ B)[\sigma] && = \Pi\ (A[\sigma])\ (B[\sigma \uparrow \cul A \cur])
  \end{alignedat}
\end{flalign*}
The definition directly follows the equations $\U[]$, $\El[]$, and $\Pi[]$ from
the syntax of regular types. The remaining equations can be proved by induction.
\begin{align*}
  \AXC{$A : \TV\ \Gamma$}\UIC{$A[\id] \equiv A$}\DP
  && \AXC{$A : \TV\ \Theta$}
  \AXC{$\sigma : \TV\ \Delta\ \Theta$}
  \AXC{$\nu : \TV\ \Gamma\ \Delta$}
  \TIC{$A[\sigma \circ \nu] \equiv A[\sigma][\nu]$}\DP
\end{align*}

Put together, this defines the evaluation function: $\U$, $\El$, and $\Pi$ are
interpreted by the respective constructors, substitutions are applied using the
previous recursive definition, the equations $\U[]$, $\El[]$, and $\Pi[]$ hold
trivially, and we just verified that $[\id]$ and $[][]$ are respected.
It is trivial to verify that this evaluation function is indeed the inverse of
the embedding, therefore regular and substitution-free types are isomorphic.

This gives an alternative, much simpler induction principle for types.
\begin{lemma}
  \label{lem:typeInduction}
  To define a function on types, it suffice to define it for the constructors
  $\U$, $\El$, and $\Pi$.
\end{lemma}
\begin{proof}
  If the function is defined on $\U$, $\El$, and $\Pi$, then it is correctly
  defined on substitution-free types. Thus it can be extended to regular types
  using the previously defined isomorphism.
\end{proof}

\subsection{Type Skeletons}
If we were to immediately define strong computability, we would face a second
issue regarding the induction principle for types: it will often be the case
that when proving a result by induction on types, and considering a type
$\Pi\ A\ B$, we need to apply the induction hypothesis not on $B$, but instead
on $B[\sigma]$ for some substitution $\sigma$, which is not allowed by the
induction principle of types.

However, if we were to forget substitutions altogether, then $B$ or $B[\sigma]$
would be the same. This is exactly the idea behind the skeleton of a type:
by deleting all substitutions, we obtain a well-founded notion of size of types,
for which $B$ and $B[\sigma]$ are equivalent.

Formally, a type skeleton correspond to the non-dependent structure of types:
either a base type or a function type.
\begin{align*}
  & \data\ \Sk : \Set\ \where \\ & \ind
  \begin{alignedat}{2}
    & \base && : \Sk \\
    & \Pi && : \Sk \to \Sk \to \Sk
  \end{alignedat}
\end{align*}
Defining the skeleton of a type is straight forward, and all quotient
constructors are clearly respected.
\begin{align*}
  & \skel : \Ty\ \Gamma \to \Sk \\ &
  \begin{alignedat}{2}
    & \skel\ \U && = \base \\
    & \skel\ (\El\ u) && = \base \\
    & \skel\ (\Pi\ A\ B) && = \Pi\ (\skel\ A)\ (\skel\ B) \\
    & \skel\ (A[\sigma]) && = \skel\ A
  \end{alignedat}
\end{align*}

Using the skeleton of types as size indicators for induction --- which formally
means proceeding by induction on the skeleton, then by pattern matching on the
type itself --- the example of problematic induction given at the beginning of
this subsection becomes valid.
\begin{lemma}
  \label{lem:typeInduction2}
  To define a function on types, it suffice to define it on the base types $\U$
  and $\El$, and to define it on any type $\Pi\ A\ B$ while assuming that it is
  defined on any type $C$ with the same skeleton as either $A$ or $B$.
\end{lemma}
\begin{proof}
  It suffice to define the function on substitution-free types by induction on
  the skeleton of the type. The function can then be extended to regular type
  using the isomorphism between regular and substitution-free types, as in
  lemma~\ref{lem:typeInduction}.
\end{proof}

\subsection{Strong Computability}
The proof of termination is based on a Tait-style~\cite{tait1967} predicate
on values, called strong computability. This subsection introduces strong
computability, together with some important lemmas.

Strong computability is defined by induction on types, as termination of $\q$
for base types, and stability by application for function types. The
generalised induction principle of lemma~\ref{lem:typeInduction2} is used.
\begin{flalign*}
  & \scv : \{A : \Ty\} \to \Val\ \Gamma\ A \to \Set \\ &
  \begin{alignedat}{3}
    & \scv_{\U}\ v && =\ && \Sigma(n : \Nf\ \Gamma\ \U)\ \q\ v \Da n \\
    & \scv_{(\El\ u)}\ v && =\ && \Sigma(n : \Nf\ \Gamma\ (\El\ u))\ \q\ v \Da n \\
    & \scv_{(\Pi\ A\ B)}\ f && =\ &&
    \forall(\alpha : \Vars\ \Delta\ \Gamma) (v : \Val\ \Delta\ A^{+\alpha}) \to \scv\ v \to \\
    & && && \Sigma(C : \Ty\ \Delta)\ \Sigma(w : \Val\ \Delta\ C) \\
    & && && (f^{+\alpha}\ @\ v \Da w) \ \land\ (\scv\ w) \ \land\ (\skel\ C \equiv \skel\ B)
  \end{alignedat}
\end{flalign*}
Some remarks can be made regarding the case of function types.
Firstly, stability under application is understood up to weakening, i.e.\
the argument $v$ need not be in the same context $\Gamma$ as the function $f$,
but may instead come from a weaker context $\Delta$, where the weakening
$\alpha : \Vars\ \Delta\ \Gamma$ expresses that $\Delta$ is weaker than $\Gamma$.

Secondly, as in the definition of the evaluation relation, we prefer not to
restrict the result type to simplify the upcoming proofs, hence we merely
require that there exist a value $w$ of some type $C$. However, the definition
would not be sound without any restriction on $C$, which is why we ask for $C$
to have the same skeleton as $B$. In this way, strong computability for
$\Pi\ A\ B$ is defined based on strong computability of types with the same
skeleton as either $A$ or $B$.

Strong computability is extended to environments pointwise.
\begin{flalign*}
  & \sce : \Env\ \Gamma\ \Delta \to \Set \\ &
  \begin{alignedat}{2}
    & \sce\ \epsilon && = \top \\
    & \sce\ (\rho,v) && = \sce\ \rho\ \land\ \scv\ v
  \end{alignedat}
\end{flalign*}

Let us now prove some lemma on strong computability. Throughout this subsection,
we implicitly use lemma~\ref{lem:typeInduction2} when proceeding by induction on
types.
\begin{lemma}
  \label{lem:scvWk}
  Strong computability is stable under weakening:
  \[
    \AXC{$v : \Val\ \Gamma\ A$}
    \AXC{$\scv\ v$}
    \AXC{$\alpha : \Vars\ \Delta\ \Gamma$}
    \TIC{$\scv\ v^{+\alpha}$}
    \DP \qquad
    \AXC{$\rho : \Env\ \Gamma\ \Theta$}
    \AXC{$\sce\ \rho$}
    \AXC{$\alpha : \Vars\ \Delta\ \Gamma$}
    \TIC{$\sce\ \rho^{+\alpha}$}
    \DP
  \]
\end{lemma}
\begin{proof}
  For values, the proof is by induction on the type. For base types, stability
  of $\q$ under weakening is used. For function types, the proof is immediate,
  since the definition of strong computability already accounts for weakening.

  For environments, the proof is trivial by induction.
\end{proof}

\begin{lemma}
  Strong computability is a mere proposition, i.e.\ any two proofs of strong
  computability are computationally equivalent.
  \label{lem:scvProp}
  \[
    \AXC{$p,q : \scv\ v$}\UIC{$p \equiv q$}\DP \qquad
    \AXC{$p,q : \sce\ \rho$}\UIC{$p \equiv q$}\DP
  \]
\end{lemma}
\begin{proof}
  For values, the proof is by induction on the type.
  For base types, we use soundness of $\q$, that is
  \[
    \AXC{$\q\ v \Da n$}
    \AXC{$\q\ v \Da m$}
    \BIC{$n \equiv m$}
    \DP
  \]
  which follows easily from lemma~\ref{lem:quoteCompl} and theorem~\ref{thm:nbe}.
  For function types, lemma~\ref{lem:evalSound} is used.

  For environments, the proof is trivial by induction.
\end{proof}

The most important lemma regarding strong computability is that it implies
termination of $\q$. A form of the converse for neutral values is proved
simultaneously.
\begin{lemma}
  \label{lem:quote}
  \[
    \AXC{$v : \Val\ \Gamma\ A$}
    \AXC{$\scv\ v$}
    \RightLabel{(quote)}
    \BIC{$\Sigma(n : \Nf\ \Gamma\ A),\ \q\ v \Da n$}
    \DP
  \]
  \[
    \AXC{$v : \NV\ \Gamma\ A$}
    \AXC{$\Sigma(n : \NN\ \Gamma\ A),\ \qn\ v \Da n$}
    \RightLabel{(unquote)}
    \BIC{$\scv\ (\neu\ v)$}
    \DP
  \]
\end{lemma}
\begin{proof}
  By mutual induction on the type $A$. The base cases are trivial by
  definition of strong computability. Consider a function type $\Pi\ A\ B$.

  For the case $(quote)$, let $f$ be a strongly computable value of type $\Pi\ A\ B$.
  Following the definition of $\q$ for function types, we need to prove that
  there exist some $v : \Val\ (\Gamma,A)\ B$ and $n : \Nf\ (\Gamma,A)\ B$ such
  that
  \[ f^{+A}\ @\ \neu\ (\var\ \vz) \Da v \quad \land \quad \q\ v \Da n \]
  In this expression, the variable $\vz$ has type $A[\pi_1 \id]$. Furthermore
  $\qn$ trivially terminates on variables, hence $(unquote)$ implies that
  $\neu\ (\var\ \vz)$ is strongly computable by induction hypothesis.
  Then by definition of strong computability $f^{+A}\ @\ \neu\ (\var\ \vz) \Da v$
  holds for some strongly computable $v$, and we may verify using
  lemma~\ref{lem:evalCompl} that $v$ has type $B$. Since $v$ is strongly
  computable of type $B$, there exist by induction hypothesis
  $n : \Nf\ (\Gamma,A)\ B$ such that $\q\ v \Da n$.
  Therefore, $\q\ f \Da (\lam\ n)$.

  Inversely, for the case $(unquote)$, assume $\qn\ f \Da n$ with
  $f : \NV\ \Gamma\ (\Pi\ A\ B)$, and let us prove that $\neu\ f$ is strongly
  computable. Let $\alpha : \Vars\ \Delta\ \Gamma$ and
  $v : \Val\ \Delta\ A^{+\alpha}$ strongly computable. Let us prove that
  $\neu\ (\app\ f^{+\alpha}\ v)$ satisfies the conditions of the definition of
  strong computability for function types.
  Firstly,
  \[ (\neu\ f^{+\alpha})\ @\ v \Da (\neu\ (\app\ f^{+\alpha}\ v)) \]
  is immediate since $f$ is neutral.
  Furthermore, by induction hypothesis $(unquote)$ and definition of $\qn$,
  to prove that $\neu\ (\app\ f^{+\alpha}\ v)$ is strongly computable, it suffice
  to check that $\qn$ terminates on $f^{+\alpha}$ and $\q$ terminates on $v$.
  The former holds by hypothesis using that $\qn$ is stable by weakening,
  while the latter holds by induction hypothesis $(quote)$.
  Finally, one may verify that the type of $\neu\ (\app\ f^{+\alpha}\ v)$ can be
  expressed as $B$ with some substitutions and weakenings applied, hence its
  skeleton is the same as $B$.
  It follows that $f$ is strongly computable.
\end{proof}

\begin{lemma}
  \label{lem:idenvsc}
  The identity environment is strongly computable.
  \[ \AXC{$\Gamma : \Con$}\UIC{$\sce\ \idenv_{\Gamma}$}\DP \]
\end{lemma}
\begin{proof}
  Lemma~\ref{lem:quote} implies that all variables are strongly computable
  because they are neutral values for which $\qn$ trivially terminates. The
  result follows by induction on $\Gamma$, using lemma~\ref{lem:scvWk}.
\end{proof}

\subsection{Termination}
All the tools are now available to prove the main termination result.
\begin{theorem}
  \label{thm:eval}
  Evaluation in a strongly computable environment terminates, and yields a
  strongly computable result.
  \[
    \AXC{$u : \Tm\ \Gamma\ A$}
    \AXC{$\rho : \Env\ \Delta\ \Gamma$}
    \AXC{$\sce\ \rho$}
    \TIC{$\Sigma(B : \Ty\ \Delta)\Sigma(v : \Val\ \Delta\ B)\ \eval\ u\ \rho \Da v\ \land\ \scv\ v$}
    \DP
  \]
  \[
    \AXC{$\sigma : \Tms\ \Gamma\ \Theta$}
    \AXC{$\rho : \Env\ \Delta\ \Gamma$}
    \AXC{$\sce\ \rho$}
    \TIC{$\Sigma(\nu : \Env\ \Gamma\ \Theta)\ \evals\ \sigma\ \rho \Da \nu\ \land\ \sce\ \nu$}
    \DP
  \]
\end{theorem}
The theorem is proved by induction on terms and substitutions. Regular
constructors are unproblematic, in the sense that the proofs does not change
significantly compared to the case of an unquotiented syntax. However, we also
need to verify that quotient constructors are respected, i.e.\ that for every
equality constructor $u \equiv v$, the proof (seen as a function) of
theorem~\ref{thm:eval} gives equal results on $u$ and $v$.

A simple way to ensure this is to prove that the types corresponding to
theorem~\ref{thm:eval} are mere propositions. In that case, when considering
an equality constructor $u \equiv v$, the result of a proof of
theorem~\ref{thm:eval} on $u$ and $v$ will necessarily be equal since both are
elements of the same mere proposition.
\begin{lemma}
  \label{lem:evalProp}
  For any $u : \Tm\ \Gamma\ A$, $\sigma : \Tms\ \Gamma\ \Theta$ and
  $\rho : \Env\ \Delta\ \Gamma$, the following types are mere propositions.
  \[ \Sigma(B : \Ty\ \Delta)\Sigma(v : \Val\ \Delta\ B)\ \eval\ u\ \rho \Da v\ \land\ \scv\ v \qquad
  \Sigma(\nu : \Env\ \Gamma\ \Theta)\ \evals\ \sigma\ \rho \Da \nu\ \land\ \sce\ \nu \]
\end{lemma}
\begin{proof}
  By lemma~\ref{lem:evalSound}, a term can only evaluate to a single value $v$.
  Furthermore, the types $\eval\ u\ \rho \Da v$ and $\scv\ v$ are mere
  propositions, by definition and by lemma~\ref{lem:scvProp} respectively.
  The result follows.

  The proof is similar in the case of substitutions.
\end{proof}

\begin{proof}[Proof of Theorem~\ref{thm:eval}]
  By induction on terms and substitutions. We split the constructors into three
  groups:
  \begin{itemize}
  \item All quotient constructors are respected by lemma~\ref{lem:evalProp}.
  \item Almost all regular constructors are very straight forward: the result of
    evaluation is obtained by following the definition of the evaluator and
    applying the induction hypotheses, and strong computability of the result
    comes directly from the hypotheses. The exceptions to this pattern are $\lam$
    and $\app$, for which we give detailed proofs below.
  \item For an abstraction $\lam u$ of type $\Pi\ A\ B$ evaluated in a strongly
    computable environment $\rho : \Env\ \Delta\ \Gamma$, evaluation is trivial
    since it simply yields the closure $\clos\ u\ \rho$. Let us show that this
    closure is strongly computable.

    Let $\alpha : \Vars\ \Theta\ \Delta$, and $v : \Val\ \Theta\ (A[\cul \rho \cur]^{+\alpha})$
    strongly computable. Then by lemma~\ref{lem:scvWk}, $(\rho^{+\alpha},v)$ is
    a strongly computable environment, hence by induction hypothesis there exists
    $w$ strongly computable such that $\eval\ u\ (\rho^{+\alpha},v) \Da w$.
    It follows that $(\clos\ u\ \rho)^{+\alpha}\ @\ v \Da w$. Finally, we
    may verify using lemma~\ref{lem:evalCompl} that the type of $w$ must have
    the same skeleton as $B$. It follows that $\clos\ u\ \rho$ is strongly
    computable.
  \item Consider an application $\app\ u$ with $u : \Tm\ \Gamma\ (\Pi\ A\ B)$
    evaluated in a strongly computable environment $(\rho,v) : \Env\ \Delta\ (\Gamma,A)$.
    By induction hypothesis, there exists $f$ strongly computable such that
    $\eval\ u\ \rho \Da f$. It can be verified using lemma~\ref{lem:evalCompl}
    that $f$ has type $\Pi\ (A[\cul \rho \cur])\ (B[\cul \rho \cur \uparrow A])$.
    Hence, because $f$ and $v$ are strongly computable, there exist $w$ strongly
    computable such that $f\ @\ v \Da w$. Then we obtain by the definition of
    the evaluation relation that $\eval\ (\app\ u)\ (\rho,v) \Da w$, proving
    the result.
  \end{itemize}
\end{proof}

\begin{proposition}[Termination]
  \label{prop:termination}
  Normalisation terminates.
  \[ \AXC{$u : \Tm\ \Gamma\ A$}\UIC{$\Sigma(n : \Nf\ \Gamma\ A),\ \norm\ u \Da n$}\DP \]
\end{proposition}
\begin{proof}
  Let $u : \Tm\ \Gamma\ A$. By lemma~\ref{lem:idenvsc} and theorem~\ref{thm:eval},
  there exist $v$ strongly computable such that $\eval\ u\ \idenv \Da v$.
  By lemma~\ref{lem:evalCompl}, one may verify that $v$ has type $A$.
  Finally, by lemma~\ref{lem:quote}, there exist $n : \Nf\ \Gamma\ A$ such that
  $\q\ v \Da n$. It follows that $\norm\ u \Da n$.
\end{proof}

By propositions~\ref{prop:soundness} and~\ref{prop:termination}, $\norm$ defines a
function from terms to normal forms, and by propositions~\ref{prop:completeness}
and~\ref{prop:stability}, it is the inverse of the embedding of normal forms.
Therefore, we have proved that big step normalisation defines a normalisation
function.

\section{Formalisation of BSN in a Cubical Type Theory}
\label{sec:cubical}
For simplicity, we have presented this paper in a strict type theory. However,
our formalisation of big step normalisation was instead done in a cubical type
theory (CTT, cf.~\cite{cchm}) using the cubical mode of Agda~\cite{norell2007agda}.
This choice allows to easily express QIIT as a special case of higher inductive
types (HIT, see~\cite{hott}), which from the technical point of view is a
notable improvement over previous implementations of QIIT in non-cubical Agda,
which had to introduce all quotient constructors as additional axioms
(e.g.~\cite{kaposi2016normalisation}).

As explained in~\cite{kaposi2016type}, simply considering a QIT as a special
case of HIT leads to unexpected results. For instance, in the case of the
quotiented syntax, $\U[]$ and $[\id]$ give two proofs of $\U[\id] \equiv \U$,
and these proofs are distinct in a non-strict type theory. Therefore, this naive
implementation of QIT leads to a syntax which is not a set in the type theoretic
sense, i.e.\ uniqueness of identity proofs (UIP) does not holds.
It follows by Hedberg's theorem~\cite{hedberg1998coherence} that equality is
undecidable in this syntax, which is definitively not what was expected.

The solution is of course to truncate the syntax to a set, by the addition of
the following constructors:
\begin{alignat*}{2}
  & \isSet\Ty && : \{A\ B : \Ty\ \Gamma\}\ (p\ q : A \equiv B) \to p \equiv q \\
  & \isSet\Tms && : \{\sigma\ \nu : \Tms\ \Gamma\ \Delta\}\ (p\ q : \sigma \equiv \nu) \to p \equiv q \\
  & \isSet\Tm && : \{s\ t : \Tm\ \Gamma\ A\}\ (p\ q : s \equiv t) \to p \equiv q
\end{alignat*}
Note that the corresponding constructor for contexts is unnecessary, because
it can be proved that contexts form a set from the assumption that types are a
family of sets.

In order to adapt the proof of big step normalisation to CTT with this new
implementation of QIIT, there are two problems to solve:
\begin{itemize}
\item The proof of BSN uses the UIP axiom of the strict type theory.\footnote{%
    While we never explicitly refer to the UIP axiom in this presentation of the
    proof, it is used whenever we prove a lemma of the form
    $L : (x : A) \to f(x) \equiv g(x)$ by induction on a QIT $A$.
    Indeed, for a quotient constructor of type $a \equiv b$ in $A$, we need to
    provide an equality between equalities $L(a) \equiv L(b)$, which is trivial
    with UIP, but is otherwise generally problematic.
    This situation occurs e.g.\ when proving the coherence lemmas for weakening
    of values.
  }
  Since we lose this axiom in CTT, its uses must be replaced.
\item The additional truncation constructors of QIIT must be taken into account
  whenever we use induction on a QIIT.
\end{itemize}
Both problems can be solved together by proving that all the types used in the
proof of BSN are in fact sets. Indeed, in that case any use of the UIP axiom
can be replaced by the proof of UIP for the appropriate type, and when defining
a function by induction on a QIIT, the set-truncation constructor can be mapped
to the proof that the codomain is a set.

Which types do we actually need to prove are sets ? QIIT are explicitly
truncated to sets, hence there is nothing to prove. Propositional types (i.e.\
in which any two elements are equal) such as the big step relations, or strong
computability, are always sets. What remains are regular inductive types, such
as variables, normal forms, substitution-free types\dots{} We will not detail
The proofs of UIP for such types, as they are fairly repetitive. Hedberg's
theorem is an extremely useful tool, as it for instance gives a simple proof
that variables form a set, using that equality of variables is decidable.
For types which do not a priori have decidable equality (e.g.\ normal forms),
it is still possible to adapt the techniques and lemmas used for Hedberg's
theorem to prove UIP.

\section{Conclusion and Further Work}
We have formalised big step normalisation for a simple dependent type theory,
and proved its correctness. Crucially, a quotiented syntax of type theory based
on QIIT is used to reduces the complexity of this proof. While the proof of BSN
for type theory shares many similarities with the case of the simply-typed
lambda-calculus, we have developed some new techniques specifically for
dependent types, for instance a simplified induction principle for the syntax
of types.

This proof is an important step for big step normalisation, as it shows that
this method is not limited to simple cases of typed lambda calculus, but can
also be generalised to dependent types. Of course, much remains to do in that
line of work. We have only proved big step normalisation in the context of a
minimal type theory, and it could be interesting to consider some extension.
The addition of dependent sum types for instance, would probably be fairly
straight forward. Introducing equality types and universes may be a more subtle
problem.

This work is also an interesting application of the QIIT syntax of type theory,
since it provides an example in which using this syntax has an important impact
on the proof. The implementation of the QIIT as HIT in cubical Agda, and its use
in the formalisation of BSN is also a practical validation of ideas which were
developed in~\cite{kaposi2016type}.

\bibliography{types.bib}
\end{document}

%%% Local Variables:
%%% mode: latex
%%% TeX-master: t
%%% End:
