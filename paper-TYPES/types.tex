\documentclass[a4paper,english,cleveref,autoref,draft]{lipics-v2019}

\usepackage{bussproofs}
\EnableBpAbbreviations

\newcommand{\agdaSymb}[1]{\mathsf{#1}}
\newcommand{\agdaKW}[1]{\mathbf{#1}}

\newcommand{\ind}{\hspace{1em}}

\newcommand{\data}{\agdaKW{data}}
\newcommand{\where}{\agdaKW{where}}
\newcommand{\Alet}{\agdaKW{let}}
\newcommand{\Ain}{\agdaKW{in}}

\newcommand{\Set}{\agdaSymb{Set}}
\newcommand{\Prop}{\agdaSymb{Prop}}
\newcommand{\Ty}{\agdaSymb{Ty}}
\newcommand{\Con}{\agdaSymb{Con}}
\newcommand{\Tms}{\agdaSymb{Tms}}
\newcommand{\Tm}{\agdaSymb{Tm}}
\newcommand{\id}{\agdaSymb{id}}
\newcommand{\app}{\agdaSymb{app}}
\newcommand{\lam}{\lambda}
\newcommand{\vz}{\agdaSymb{vz}}
\newcommand{\vs}{\agdaSymb{vs}}
\newcommand{\Var}{\agdaSymb{Var}}
\newcommand{\Vars}{\agdaSymb{Vars}}
\newcommand{\wk}{\agdaSymb{wk}}
\newcommand{\Ne}{\agdaSymb{Ne}}
\newcommand{\Val}{\agdaSymb{Val}}
\newcommand{\Env}{\agdaSymb{Env}}
\newcommand{\NV}{\Ne\ \Val}
\newcommand{\var}{\agdaSymb{var}}
\newcommand{\neu}{\agdaSymb{neu}}
\newcommand{\clos}{\agdaSymb{clos}}
\newcommand{\qVal}{\agdaSymb{qVal}}
\newcommand{\qEnv}{\agdaSymb{qEnv}}
\newcommand{\idenv}{\agdaSymb{idenv}}
\newcommand{\Nf}{\agdaSymb{Nf}}
\newcommand{\NN}{\Ne\ \Nf}
\newcommand{\eval}{\agdaSymb{eval}}
\newcommand{\evals}{\agdaSymb{evals}}
\newcommand{\q}{\agdaSymb{quote}}
\newcommand{\qn}{\agdaSymb{quoten}}
\newcommand{\norm}{\agdaSymb{norm}}
\newcommand{\scv}{\agdaSymb{scv}}
\newcommand{\sce}{\agdaSymb{sce}}

\newcommand{\cul}{\ulcorner}
\newcommand{\cur}{\urcorner}
\newcommand{\Ra}{\Rightarrow}
\newcommand{\Da}{\Downarrow}

\newcommand{\Beq}{\simeq_{\beta\eta}}


\bibliographystyle{plainurl} % Mandatory style

\title{Big Step Normalisation for Type Theory}

\author{Thorsten Altenkirch}
{School for Computer Science, University of Nottingham, United Kingdom}
{txa@cs.nott.ac.uk}{}{}

\author{Colin Geniet}
{Departement of Informatics, ENS Paris-Saclay, France}
{colin.geniet@ens-paris-saclay.fr}{}{}

\authorrunning{T. Altenkirch and C. Geniet}

\Copyright{Thorsten Altenkirch and Colin Geniet}

\ccsdesc[100]{General and reference~General literature}
\ccsdesc[100]{General and reference}
% TODO mandatory: Please choose ACM 2012 classifications from https://dl.acm.org/ccs/ccs_flat.cfm 

\keywords{Normalisation, lambda-calculus, type theory}

\supplement{}%TODO

%Editor-only macros:: begin (do not touch as author)%%%%%%%%%%%%%%%%%%%%%%%%%%%%%%%%%%
\EventEditors{John Q. Open and Joan R. Access}
\EventNoEds{2}
\EventLongTitle{42nd Conference on Very Important Topics (CVIT 2016)}
\EventShortTitle{CVIT 2016}
\EventAcronym{CVIT}
\EventYear{2016}
\EventDate{December 24--27, 2016}
\EventLocation{Little Whinging, United Kingdom}
\EventLogo{}
\SeriesVolume{42}
\ArticleNo{23}
%%%%%%%%%%%%%%%%%%%%%%%%%%%%%%%%%%%%%%%%%%%%%%%%%%%%%%

\begin{document}
\maketitle

\begin{abstract}
TODO
\end{abstract}

\section{Introduction}
\subsection{Normalisation}
In the context of the simply-typed lambda-calculus, let $\Tm\ \Gamma\ A$ denote
the set of terms of type $A$ in context $\Gamma$. A normalisation function can be
defined as a family of sets $\Nf\ \Gamma\ A$ of \emph{normal forms} of type
$A$ in context $\Gamma$, with computable maps
\begin{align}
  \cul \_ \cur &:\ \Nf\ \Gamma\ A → \Tm\ \Gamma\ A \tag{embedding} \\
  \norm        &:\ \Tm\ \Gamma\ A → \Nf\ \Gamma\ A \tag{normalisation}
\end{align}
satisfying for all terms $u,v$ and normal form $n$
\begin{align}
  u \Beq v \ &\Rightarrow \ \norm u \equiv \norm v \tag{soundness} \\
  u &\Beq \cul \norm\ u \cur \tag{completeness} \\
  n &\equiv \norm\ \cul n \cur  \tag{stability}
\end{align}
where $\Beq$ denotes the $\beta\eta$-equivalence relation on terms.

The traditional way of defining normalisation for the typed lambda calculus is
through rewriting theory. The $\beta\eta$-reduction relation is proved to be
confluent and strongly normalising. Normal forms are normal forms in the sense
of rewriting, i.e.\ terms which can not be reduced. Normalisation is obtained by
reducing a term until a normal form is reached. Termination and confluence
ensure that the definition is correct, completeness and stability are immediate,
while soundness follows from confluence. Cf.~\cite{girard1989proofs} for a
detailed proof of this result for the simply-typed lambda-calculus and some of
its variants (System T, System F).

Unfortunately, this method fails for many other variants of lambda-calculi.
E.g.\ the lambda-calculus with explicit substitutions is not strongly
normalising~\cite{mellies1995explicit}, and the lambda-calculus with
coproduct types (i.e.\ disjoint union) is not confluent~\cite{dougherty1995coproducts}.

A more modern approach is \emph{normalisation by evaluation}~(NBE), introduced
by Berger and Schwichtenberg~\cite{berger1991inverse} for the simply-typed
lambda-calculus. The idea is to interpret terms into a semantic models,
meaning that abstractions are interpreted by actual functions.
A map from the model into normal forms is then defined, giving rise to the
normalisation function. 
This method was for instance used to prove decidability of equivalence for the
lambda-calculus with coproducts~\cite{altenkirch2001normalization}.

In~\cite{chapman2006tait}~\cite{chapman2009bsn}, J.~Chapman proposes
\emph{big step normalisation}~(BSN) as yet another normalisation method.
Unlike NBE, normalisation is here defined through a purely syntactic evaluator.
This notably has the advantage that the the normalisation function enjoys a
relatively simple definition.

\subsection{BSN for the simple lambda-calculus}
This section summarises BSN for the simply-typed lambda-calculus, as presented
in~\cite{chapman2009bsn}.

The normalisation function is defined in two parts. Firstly, terms are
evaluated in an environment of values, yielding a value. This evaluator
is defined mutually with an application function, which computes the application
of a function value of type $A \Ra B$ to an argument value of type $A$.
Finally, values are normalised to normal forms.
Normalisation of terms is done by evaluating in the identity environment,
and normalising the resulting value.

Values are purely syntactic, and are very similar to the notion of values used
in programming languages, notably terms are not evaluated under lambdas.

Evaluation and normalisation both have fairly simple definition, but are
not structurally recursive, meaning that their termination is not obvious.
For that purpose, a predicate called \emph{strong computability} is defined
on values as
\begin{itemize}
\item A value $v$ of base type is strongly computable if normalisation
  terminates on $v$.
\item A value $f$ of type $A \Ra B$ is strongly computable if for any $v$ strongly
  computable of type $A$, the application of $f$ to $v$ terminates and the
  result is strongly computable.
\end{itemize}

The following results can then be proved.
\begin{itemize}
\item Normalisation terminates on any strongly computable value,
  and any \emph{neutral} value on which normalisation terminates
  is strongly computable.
\item Evaluation of a term in a strongly computable environment terminates
  and the result is strongly computable.
\end{itemize}
Termination follows from those results.

The proof of completeness is simple, and stability is proved by induction
on values and normal forms.
Soundness is more complicated, and requires a method similar to strong
computability, only this time with a binary relation on values, defined as
equality of normal forms at the base type.

\subsection{BSN for type theory}
J.~Chapman also considered BSN for dependent type theory
in~\cite{chapman2009type}, but did not provide a proof of correctness,
due in particular to the high complexity of working with the (unquotiented)
syntax of type theory in a formal setting.

We propose a few methods to simplify the previous proof in the case of type
theory, allowing us to complete it in the theorem prover Agda.
One of the most important change is the use of a quotiented syntax for type
theory based on~\cite{kaposi2016type}.


\section{Quotiented Syntax of the Lambda-Calculus}
We use a quotiented syntax for the lambda-calculus, based on~\cite{kaposi2016type}
and~\cite{kaposi2016normalisation}. It uses intrinsic typing, explicit substitutions,
De Bruijn indices and categorical constructors.

Types and typing contexts are defined as simple inductive types.
The abstract base type is denoted by $\iota$.
Due to the use of De Bruijn indices, contexts are simply lists of types.
We denote types by $A,B,C,\dots$, and contexts by $\Gamma,\Delta,\Theta,\Psi,\dots$.
\begin{alignat*}{2}
  & \data\ \Ty : \Set\ \where && \data\ \Con : \Set\ \where \\
  & \ind
  \begin{alignedat}{2}
    & \iota && : \Ty \\
    & \_\Ra\_ && : \Ty \to \Ty \to \Ty \qquad
  \end{alignedat}
  && \ind
  \begin{alignedat}{2}
    & \bullet && : \Con \\
    & \_,\_ && : \Con \to \Ty \to \Con
  \end{alignedat}
\end{alignat*}

Terms $\Tm\ \Gamma\ A$ and substitutions $\Tms\ \Gamma\ \Delta$ are defined mutually
as quotient inductive types (QIT, cf.~\cite{kaposi2016type}).
\begin{alignat*}{2}
  & \data\ \Tms && : \Con \to \Con \to \Set \\
  & \data\ \Tm && : \Con \to \Ty \to \Set
\end{alignat*}
The constructors can be summarised as follows.
\begin{itemize}
\item Contexts and substitutions form a category, with terminal object $\bullet$.
\item Context extension $\_,\_$ can be seen as selected cartesian products.
\item Application of a substitution is an explicit constructor, denoted
  by $\_[\_]$.
\item Lambda-abstraction and application form a natural isomorphism
  between $\Tm\ (\Gamma , A)\ B$ and $\Tm\ \Gamma\ (A \Ra B)$,
  corresponding to a categorical exponential.
\end{itemize}
We denote terms by $u,s,t,\dots$ and substitutions by $\sigma,\nu,\delta,\dots$.
The constructors are listed below, with regular constructors on the left
and equality constructors on the right.
\begin{alignat*}{2}
  & \data\ \Tms\ \where && \data \Tms\ \where \\
  & \ind
  \begin{alignedat}{2}
    & \id       && : \Tms\ \Gamma\ \Gamma \\
    & \_\circ\_ && : \Tms\ \Delta\ \Theta \to \Tms\ \Gamma\ \Delta\ \to \Tms\ \Gamma\ \Theta \\
    & \epsilon  && : \Tms\ \Gamma\ \bullet \\
    & \_,\_     && : \Tms\ \Gamma\ \Delta \to \Tm \Gamma\ A \to \Tms\ \Gamma\ (\Delta,A) \qquad \\
    & \pi_1     && : \Tms\ \Gamma\ (\Delta,A) \to \Tms\ \Gamma\ \Delta \\
    & && \\ & &&
  \end{alignedat}
  && \ind
  \begin{alignedat}{2}
    & \id\circ && : \id \circ \sigma \equiv \sigma \\
    & \circ\id && : \sigma \circ \id \equiv \sigma \\
    & \circ\circ && : (\sigma \circ \nu) \circ \delta \equiv \sigma \circ (\nu \circ \delta) \\
    & \epsilon\eta && : \{\sigma : \Tms\ \Gamma\ \bullet\} \to \sigma \equiv \epsilon \\
    & \pi_1\beta && : \pi_1(\sigma,u) \equiv \sigma \\
    & \pi\eta  && : \pi_1\,\sigma , \pi_2\,\sigma \equiv \sigma \\
    & ,\circ  && : (\sigma,u) \circ \nu \equiv (\sigma \circ \nu),(u[\nu])
  \end{alignedat} \\
  & \data\ \Tm\ \where && \data\ \Tm\ \where \\
  & \ind
  \begin{alignedat}{2}
    & \pi_2     && : \Tms\ \Gamma\ (\Delta,A) \to \Tm\ \Gamma\ A \\
    & \_[\_]    && : \Tm\ \Delta\ A \to \Tms\ \Gamma\ \Delta \to \Tm\ \Gamma\ A \\
    & \lam      && : \Tm\ (\Gamma,A)\ B \to \Tm\ \Gamma\ (A \Ra B) \\
    & \app      && : \Tm\ \Gamma\ (A \Ra B) \to \Tm\ (\Gamma,A)\ B
  \end{alignedat}
  && \ind
  \begin{alignedat}{2}
    & \pi_2\beta && : \pi_2(\sigma,u) \equiv u \\
    & \beta    && : \app\,t (\lam u) \equiv u \\
    & \eta     && : \lam (\app\, u) \equiv u \\
    & \lam[]   && : (\lam u)[\sigma] \equiv \lam(u[\sigma\uparrow A])
  \end{alignedat}
\end{alignat*}
The last equality constructor uses the lifting of a substitution by a type,
defined as
\begin{alignat*}{1}
  & \_\uparrow\_      : \Tms\ \Gamma\ \Delta \to (A:\Ty) \to \Tms\ (\Gamma,A)\ (\Delta,A) \\
  & \sigma \uparrow A = (\sigma \circ \pi_1\,\id) , \pi_2\,\id
\end{alignat*}

The usual application can also be obtained from this syntax.
We first define the substitution of a single term $<\_>$.
Application is obtained by combining it with $\app$.
\begin{alignat*}{2}
  & <u> && = \id , u \\
  & \_\$\_ && : \Tm\ \Gamma\ (A \Ra B) \to \Tm\ \Gamma\ A \to \Tm\ \Gamma\ B \\
  & f\ \$\ u && = (\app\, f)[<u>]
\end{alignat*}


\section{Weakening}
We now introduce the notion of weakening, which is important in later
constructions and proofs.

Several definition of weakening may be considered, our experience is that any
sensible notion of weakening should be appropriate if contexts and weakenings
form a category. For instance, in~\cite{chapman2009normalisation}, J.~Chapman
uses \emph{order preserving embeddings} (see also~\cite{altenkirch1995categorical}).
We prefer to use the slightly more general notion of \emph{renamings}, which are
simply lists of variables, because it is easier to adapt to type theory
(cf.~\cite{kaposi2016normalisation}).

Variables are defined as De Bruijn indices, and renamings as list of variables.
We also define the embeddings of variables into terms.
\begin{alignat*}{2}
  & \data\ \Var : \Set\ \where && \data\ \Vars : \Set\ \where \\ & \ind
  \begin{alignedat}{2}
    & \vz && : \Var\ (\Gamma,A)\ A \\
    & \vs && : \Var\ \Gamma\ A \to \Var\ (\Gamma,B)\ A \qquad
  \end{alignedat} && \ind
  \begin{alignedat}{2}
    & \epsilon && : \Vars\ \Gamma\ \bullet \\
    & \_,\_ && : \Vars\ \Gamma\ \Delta \to \Var \Gamma\ A \to \Vars\ \Gamma\ (\Delta,A)
  \end{alignedat} \\ &
  \begin{alignedat}{2}
    & \cul\_\cur && : \Var\ \Gamma\ A \to \Tm\ \Gamma\ A \\
    & \cul \vz \cur && = \pi_2\,\id \\
    & \cul \vs\,x \cur && = \cul x \cur [\pi_1\,\id]
  \end{alignedat} &&
  \begin{alignedat}{2}
    & \cul\_\cur  && : \Vars\ \Gamma\ \Delta \to \Tms\ \Gamma\ \Delta \\
    & \cul \epsilon \cur && = \epsilon \\
    & \cul \sigma,x \cur && = \cul \sigma \cur , \cul x \cur
  \end{alignedat}
\end{alignat*}

Definitions of identity and composition require some appropriate auxiliary
functions. The identity uses the weakening $\wk$ of a renaming by a type,
and composition uses the application $\_[\_]$ of a renaming to a variable.
We omit the definitions, which are simple.
\begin{alignat*}{2}
  & \wk && : (A : \Ty) \to \Vars\ \Gamma\ \Delta \to \Vars\ (\Gamma,A)\ \Delta \\
  & \id && : \Vars\ \Gamma\ \Gamma \\
  & \_[\_] && : \Var\ \Delta\ A \to \Vars\ \Gamma\ \Delta \to \Var\ \Gamma\ A \\
  & \_\circ\_ && : \Vars\ \Delta\ \Theta \to \Vars\ \Gamma\ \Delta \to \Vars\ \Gamma\ \Theta
\end{alignat*}
It is easy to show that composition and identity of renamings respect
the categorical laws, and commute with embeddings.

Weakenings of terms and substitution are defined using embeddings,
and can be proved to form presheaves over the category of renamings.
\begin{alignat*}{4}
  & \_\ ^{+\_} && : \Tm\ \Delta\ A \to \Vars\ \Gamma\ \Delta \to \Tm\ \Gamma\ A \qquad
  && \_\ ^{+\_} && : \Tms\ \Delta\ \Theta \to \Vars\ \Gamma\ \Delta \to \Tms\ \Gamma\ \Theta \\
  & u^{+\delta} && = u [\cul\delta\cur]
  && \sigma^{+\delta} && = \sigma \circ \delta \\
  & +\id && : u^{+\id} \equiv u
  && +\id && : \sigma^{+\id} \equiv \sigma \\
  & +\circ && : u^{+ (\delta \circ \nu)} \equiv (u^{+\delta})^{+\nu}
  && +\circ && : \sigma^{+ (\delta \circ \nu)} \equiv (\sigma^{+\delta})^{+\nu}
\end{alignat*}

This will be a general pattern in later proofs: every family of sets that will
be defined (e.g.\ values, normal forms) form a presheaf over the category of
renamings, which simply means that the elements can be weakened coherently.
Similarly, functions are natural transformations, i.e.\ are compatible
with weakening, and predicates are sub-presheaves, meaning that they are stable
under weakening.

The corresponding proofs are all straight forward, and we will generally not
mention them. We abuse notations and use $\_\ ^{+\_}$ to denote all weakenings.

Given a type $A$, one can consider $\wk\ A\ \id : \Vars\ (\Gamma,A)\ \Gamma$,
the weakening of a context $\Gamma$ by~$A$.
We abuse notations and write $u^{+A}$ for $u^{+ (\wk\ A\ \id)}$.


\section{BSN for the Quotiented Lambda Calculus}
In this section, we adapt the definition of big step normalisation
from~\cite{chapman2009bsn} to our quotiented syntax.
We first define values and normal forms, and then the normalisation function.

\subsection{Values}
A value is either a closure, corresponding to the delayed evaluation of a
lambda-abstraction, or a neutral value, that is the stuck application of a
variable to values. This suggests the following definition.

We first define a general notion of neutral terms: if $X$ is a subset of terms,
a neutral $X$ is a variable applied to any number of $X$.
\begin{alignat*}{1}
  & \data\ \Ne\ (X : \Con \to \Ty \to \Set) : \Con \to \Ty \to \Set\ \where \\ & \ind
  \begin{alignedat}{2}
    & \var && : \Var\ \Gamma\ A \to \Ne\ X\ \Gamma\ A \\
    & \app && : \Ne\ X\ \Gamma\ (A \Ra B) \to X\ \Gamma\ A \to \Ne\ X\ \Gamma\ B
  \end{alignedat}
\end{alignat*}

Values and environment are then mutually defined.
\begin{alignat*}{1}
  & \data\ \Val : \Con \to \Ty \to \Set\ \where \\ & \ind
  \begin{alignedat}{2}
    & \clos  && : \Tm\ (\Delta,A)\ B \to \Env\ \Gamma\ \Delta \to \Val\ \Gamma\ (A \Ra B) \\
    & \neu   && : \NV\ \Gamma\ A \to \Val\ \Gamma\ A
  \end{alignedat} \\
  & \data\ \Env : \Con \to \Con \to \Set\ \where \\ & \ind
  \begin{alignedat}{2}
    & \epsilon && : \Env\ \Gamma\ \bullet \\
    & \_,\_ && : \Env\ \Gamma\ \Delta \to \Val\ \Gamma\ A \to \Env\ \Gamma\ (\Delta,A)
  \end{alignedat}
\end{alignat*}
Finally, there are obvious embeddings of values and environments into terms and
substitutions, denoted by $\cul\_\cur$.

This definition would be used with an unquotiented syntax of the lambda-calculus,
as in~\cite{chapman2009bsn}, but is inappropriate with the quotiented syntax,
because some values are equivalent as terms (formally, have equal embeddings),
but are not equal.
For instance, if one considers a closure $\clos\ u\ \rho$ in which the body
$u$ never refers to the environment $\rho$, then modifying $\rho$ will yield
a distinct, yet equivalent value.
This issue means that the evaluation function may send equivalent terms to
distinct values, hence can not be defined on the quotiented syntax.

To avoid this, we introduce an additional constructor which quotients values
by the term equivalence relation, ensuring that the embedding of values is
injective.
\begin{alignat*}{1}
  & \data\ \Val\ \where \\
  & \ind \qVal : (v\ w : \Val\ \Gamma\ A) \to \cul v \cur \equiv \cul w \cur \to v \equiv w
\end{alignat*}
The corresponding result for environment can be proved by induction.
\[ \qEnv : (\rho\ \nu : \Env\ \Gamma\ \Delta) \to \cul \rho \cur \equiv \cul \nu \cur \to \rho \equiv \nu \]

Finally, weakenings of values, neutral values and environments are defined
inductively. We omit the definitions and the associated lemmas.
\begin{alignat*}{2}
  & \_\ ^{+\_} && : \Val\ \Delta\ A \to \Vars\ \Gamma\ \Delta \to \Val\ \Gamma\ A \\
  & \_\ ^{+\_} && : \NV\ \Delta\ A \to \Vars\ \Gamma\ \Delta \to \NV\ \Gamma\ A \\
  & \_\ ^{+\_} && : \Env\ \Delta\ \Theta \to \Vars\ \Gamma\ \Delta \to \Env\ \Gamma\ \Theta
\end{alignat*}
The identity environment can then be defined.
\begin{alignat*}{1}
  & \idenv : \{\Gamma : \Con\} \to \Env\ \Gamma\ \Gamma \\ &
  \begin{alignedat}{2}
    & \idenv\ \{\bullet\} && = \epsilon \\
    & \idenv\ \{\Gamma,A\} && = (\idenv\ \{\Gamma\})^{+A}, \neu\,(\var\ z)
  \end{alignedat}
\end{alignat*}

\subsection{Normal Forms}
Big step normalisation uses the notion of \emph{$\eta$-long $\beta$-normal forms},
which is also used in normalisation by evaluation
(cf.~\cite{altenkirch1995categorical} and~\cite{kaposi2016normalisation}).
\begin{alignat*}{1}
  & \data\ \Nf : \Con \to \Ty \to \Set\ \where \\ & \ind
  \begin{alignedat}{2}
    & \lam && : \Nf\ (\Gamma,A)\ B \to \Nf\ \Gamma (A \Ra B) \\
    & \neu && : \NN\ \Gamma\ \iota \to \Nf\ \Gamma\ \iota
  \end{alignedat}
\end{alignat*}
Importantly, a neutral normal form is normal only if it is of the base type.
This ensures that normal forms are sufficiently $\eta$-expended.

We omit the definitions of weakening and embedding of normal forms and neutral
normal forms, which are straight-forward inductive definitions.

\subsection{Evaluation}
The first stage of normalisation evaluates terms in an environment, yielding
a value as result.
It consists of three mutually defined functions: $\eval$ and $\evals$
evaluate terms and substitutions respectively in an environment,
while $\_@\_$ computes the application of a value to another.
\begin{alignat*}{4}
  & \evals\ \id\ \rho && = \rho
  && (\clos\ u\ \rho)\ @\ v && = \eval\ u\ (\rho,v) \\
  & \evals\ (\sigma \circ \nu)\ \rho && = \evals\ \sigma\ (\evals\ \nu\ \rho)
  && (\neu\ f)\ @\ v && = \neu\ (\app\ f\ v) \\
  & \evals\ \epsilon\ \rho && = \epsilon \\
  & \evals\ (\sigma,u)\ \rho && = (\evals\ \sigma\ \rho) , (\eval\ u\ \rho) \\
  & \evals\ (\pi_1\, \sigma)\ \rho && = \Alet\ (\nu,v) = \evals\ \sigma\ \rho\ \Ain\ \nu \qquad &\\
  & \eval\ (\pi_2\, \sigma)\ \rho && = \Alet\ (\nu,v) = \evals\ \sigma\ \rho\ \Ain\ v \\
  & \eval\ (u[\sigma])\ \rho && = \eval\ u\ (\evals\ \sigma\ \rho) \\
  & \eval\ (\lam u)\ \rho && = \clos\ u\ \rho \\
  & \eval\ (\app\ u)\ (\rho,v) && = (\eval\ u\ \rho)\ @\ v
\end{alignat*}

Most cases are very straight forward. Note how evaluation of a lambda does
nothing, and merely returns a closure, delaying the evaluation of the body.
The latter occurs in the first case of $\_@\_$, as the application of
a closure to a value is computed by evaluating the body of the closure in the
extended environment.
In the case of the projections $\pi_1$ and $\pi_2$, the environment $\evals\ \sigma\ \rho$
must by typing be a non-empty list, and pattern matching is used to project out
the appropriate component.

The problem with this definition is that it is not structurally recursive,
and thus it is by no mean clear that the evaluator terminates.
To formally work with this a priori non-terminating function, we redefine
it as a relation between inputs and outputs. For instance, we denote by
$\eval\ t\ \rho \Da v$ the proposition `$t$ evaluates to $v$ in environment $\rho$'.
\begin{alignat*}{2}
  & \data\ \eval\_\,\_\Da\_ && : \Tm\ \Delta\ A \to \Env\ \Gamma\ \Delta
                                 \to \Val\ \Gamma\ A \to \Prop \\
  & \data\ \evals\_\,\_\Da\_ && : \Tms\ \Delta\ \Theta \to \Env\ \Gamma\ \Delta
                                  \to \Env\ \Gamma\ \Theta \to \Prop \\
  & \data\ \_@\_\Da\_ && : \Val\ \Gamma\ (A \Ra B) \to \Val\ \Gamma\ A \to \Val\ \Gamma\ B \to \Prop
\end{alignat*}
As an example, we define the relation corresponding to $\eval$.
The others are similar.
\begin{alignat*}{1}
  & \data\ \eval\_\,\_\Da\_\ \where \\ & \ind
  \begin{alignedat}{2}
    & \eval\pi_2 && : \evals\ \sigma\ \rho \Da (\nu,v) \to \eval\ (\pi_2\ \sigma)\ \rho \Da v \\
    & \eval[]    && : \evals\ \sigma\ \rho \Da \nu \to \eval\ u\ \nu \Da v \to \eval\ (u[\sigma])\ \rho \Da v \\
    & \eval\lam  && : \eval\ (\lam u)\ \rho \Da (\clos\ u\ \rho) \\
    & \eval\app  && : \eval\ f\ \rho \Da g \to g\ @\ v \Da w \to \eval\ (\app\ f)\ (\rho,v) \Da w
  \end{alignedat}
\end{alignat*}

Note that those relations are truncated to propositions. Form a technical point
of view, this is done by introducing truncation constructors such as
\[ \eval\Prop : (x\ y : \eval\ u\ \rho \Da v) \to x \equiv y \]

The following coherence lemmas are easily proved by induction on the evaluation
relation.

\begin{lemma}
  \label{lem:evalCompl}
  \[
    \AXC{$\eval\ u\ \rho \Da v$}
    \UIC{$\cul v \cur \equiv u[ \cul\rho\cur ]$}
    \DP \quad
    \AXC{$\evals\ \sigma\ \rho \Da \nu$}
    \UIC{$\cul\nu\cur \equiv \sigma \circ \cul\rho\cur$}
    \DP \quad
    \AXC{$f\ @\ v \Da w$}
    \UIC{$\cul f \cur\ \$\ \cul v \cur \equiv \cul w \cur$}
    \DP
  \]
\end{lemma}
Some soundness properties easily follow.
\begin{lemma}
  \label{lem:evalSound}
  \[
    \AXC{$\eval\ u\ \rho \Da v$}
    \AXC{$\eval\ u\ \rho \Da w$}
    \BIC{$v \equiv w$}
    \DP \quad
    \AXC{$\evals\ \sigma\ \rho \Da \nu$}
    \AXC{$\evals\ \sigma\ \rho \Da \delta$}
    \BIC{$\nu \equiv \delta$}
    \DP
  \]
  \[
    \AXC{$f\ @\ u \Da v$}
    \AXC{$f\ @\ u \Da w$}
    \BIC{$v \equiv w$}
    \DP
  \]
\end{lemma}
\begin{proof}
  Simple using lemma~\ref{lem:evalCompl}, and the fact that embedding of values and
  environments are injective.
\end{proof}

\subsection{Quote and Normalisation}
The second stage of normalisation, called $\q$, maps values to normal forms.
It is defined by induction on the type of the value, together with $\qn$
which maps neutral values to neutral normal forms by recursively applying $\q$.
As for the evaluator, we first give the informal definition as a function
whose termination is unclear.
\begin{alignat*}{1}
  & \q : \{A : \Ty\} \to \Val\ \Gamma\ A \to \Nf\ \Gamma\ A \\ &
  \begin{alignedat}{2}
    & \q\ \{\iota\}\ (\neu\ v) && = \neu\ (\qn\ v) \\
    & \q\ \{A \Ra B\}\ f       && = \lam (\q\ (f^{+A}\ @\ \neu\ (\var\ \vz)))
  \end{alignedat} \\
  & \qn : \NV\ \Gamma\ A \to \NN\ \Gamma\ A \\ &
  \begin{alignedat}{2}
    & \qn\ (\var\ x) && = \var\ x \\
    & \qn\ (\app\ f\ v) && = \app\ (\qn\ f)\ (\q\ v)
  \end{alignedat}
\end{alignat*}

A value of type $\iota$ is necessarily neutral, and $\q$ simply uses
$\qn$ in that case. For function types, the definition of normal forms
requires the result to be an abstraction. This is done by $\eta$-expending the
value, and applying $\q$ to the body of the resulting abstraction.
The $\eta$-expansion is somewhat technical to define. First, the function is
weakened ($f^{+A}$) to allow the introduction of a new variable represented
by the De Bruijn index $\vz$. This variable is turned into a value by the
$\var$ and $\neu$ constructors, and the weakened function is applied
using $@$, giving the body of the $\eta$-expansion.

Like evaluation, $\q$ is formally defined as a relation with the
following type.
\begin{alignat*}{1}
  & \data\ \q\_\Da\_ : \Val\ \Gamma\ A \to \Nf\ \Gamma\ A \to \Prop \\
  & \data\ \qn\_\Da\_ : \NV\ \Gamma\ A \to \NN\ \Gamma\ A \to \Prop
\end{alignat*}
A coherence lemma similar to lemma~\ref{lem:evalCompl} is proved by induction
on the relation.
\begin{lemma}
  \label{lem:quoteCompl}
  \[
    \AXC{$\q\ v \Da n$}
    \UIC{$\cul n \cur \equiv \cul v \cur$}
    \DP \quad
    \AXC{$\qn\ v \Da n$}
    \UIC{$\cul n \cur \equiv \cul v \cur$}
    \DP
  \]
\end{lemma}

Normalisation of terms is obtained by evaluating in the identity environment,
then applying $\q$.
\[ \norm\ u\ \Da\ n = \Sigma(v : \Val\ \Gamma\ A)\ \eval\ u\ \idenv \Da v \ \land \ \q\ v \Da n \]

With this definition, we can already prove stability and completeness of BSN.
\begin{theorem}[Completeness]
  \label{thm:compl}
  \[
    \AXC{$\norm\ u \Da n$}
    \UIC{$\cul n \cur \equiv u$}
    \DP
  \]
\end{theorem}
\begin{proof}
  Immediate by lemmas~\ref{lem:evalCompl} and~\ref{lem:quoteCompl}.
\end{proof}
\begin{theorem}[Stability]
  \[
    \AXC{$n : \Nf\ \Gamma\ A$}
    \UIC{$\norm\ \cul n \cur \Da n$}
    \DP\quad
    \AXC{$n : \NN\ \Gamma\ A$}
    \UIC{$\Sigma(v : \NV\ \Gamma\ A)\ \eval\ \cul n \cur \Da (\neu\ v) \ \land \ \qn\ v \Da n$}
    \DP
  \]
\end{theorem}
\begin{proof}
  By simultaneous induction on normal forms and neutral normal forms.
\end{proof}

\section{Correction of BSN}
Two main results must be proved in order to establish the correction of BSN as
previously defined.
Termination states that the normalisation relation is defined on every term.
\[ \forall (u : \Tm\ \Gamma\ A),\ \exists (n : \Nf\ \Gamma\ A),\ \norm\ u \Da n \]
Soundness states that normalisation can only give one result for each term.
\[
  \AXC{$\norm\ u \Da n$}
  \AXC{$\norm\ u \Da m$}
  \BIC{$n \equiv m$}
  \DP
\]
Termination and soundness together imply that the normalisation relation defines
a function from terms to normal forms, and the remaining correction properties
(completeness and stability) have already been proved in the previous section.

\subsection{Soundness, for free}
The original proof by J.~Chapman proves soundness using a logical relation,
similar to the arguments used to prove termination in the next section.
Unfortunately, this proof seems hard to adapt to the quotiented syntax, and
is also quite long --- remember that we are trying to reduce the size of the
proof of BSN, in order to be able to adapt it to type theory.

However, there is an other, much shorter, if not as interesting, way to prove
soundness. Recall that $\eta$-long $\beta$-normal forms are used both for NBE
and BSN. From the fact that NBE is a normalisation function, we obtain easily
that the embedding of normal forms is injective.
\begin{lemma}
  \label{lem:nfembinj}
  \[
    \AXC{$\cul n \cur \equiv \cul m \cur$}
    \UIC{$n \equiv m$}
    \DP
  \]
\end{lemma}
Using completeness, it is easy to prove soundness up to embedding.
\[
  \AXC{$\norm\ u \Da n$}
  \AXC{$\norm\ u \Da m$}
  \BIC{$\cul n \cur \equiv \cul m \cur$}
  \DP
\]
Therefore, lemma~\ref{lem:nfembinj} immediately implies soundness.

A more abstract way to understand that proof is that, since NBE and BSN are two
normalisation functions with the same notion of normal form, they are both
inverses of the embedding of normal forms, hence are equal. Thus, as soon as we
prove that BSN is indeed the inverse of embedding, all the correction properties
which are known to hold for NBE automatically hold for BSN.

One may of course argue that a proof of correction for a normalisation function
which relies on another normalisation function is somewhat pointless.
However, it should not be forgotten that the main advantage of BSN lies in its
simple definition. The proof of BSN itself on the other hand does seems to have
little advantage other the proof of NBE. As such, we think that BSN would be
most interesting as an alternative normalisation methods for variants of the
lambda-calculus for which a normalisation theorem is already established, and
not as a way to prove normalisation.

The fact that BSN and NBE define the same function only strengthens this point,
as one may see BSN not so much as a normalisation function in itself, but rather
as an alternative, syntactic definition for the function which is usually
defined through NBE.

\subsection{Termination}
Our proof of termination follows closely the original one by J.~Chapman.
The one difference is that we need to prove that our proof of the
theorem~\ref{thm:eval} respects the quotient constructors of the syntax,
which we do in one go by proving that strong computability is a mere proposition
in lemma~\ref{lem:scvProp}.

We introduce a predicate on values called strong computability.
It is defined as termination of $\q$ at the base type,
and stability by application for function types.
\begin{alignat*}{1}
  & \scv : \{A : \Ty\} \to \Val\ \Gamma\ A \to \Set \\ &
  \begin{alignedat}{3}
    & \scv\ \{\iota\}\ v && =\ && \Sigma(n : \Nf\ \Gamma\ A)\ \q\ v \Da n \\
    & \scv\ \{A \Ra B\}\ f && =\ &&
    \forall(\sigma : \Vars\ \Delta\ \Gamma) (v : \Val\ \Delta\ A) \to \scv\ v \to \\
    & && && \Sigma(w : \Val\ \Delta\ B)\ f\ @\ v \Da w \ \land\ \scv\ w
  \end{alignedat}
\end{alignat*}

Note that in the case of function type, stability under application is up to
weakening, that is the argument $v$ need not be in the same context $\Gamma$
as the function $f$, but may instead come from a weaker context $\Delta$,
where the weakening $\sigma : \Vars\ \Delta\ \Gamma$ expresses that $\Delta$
is weaker than $\Gamma$.

Strong computability is stable under weakening.
\begin{lemma}
  \label{lem:scvWk}
  \[
    \AXC{$v : \Val\ \Gamma\ A$}
    \AXC{$\scv\ v$}
    \AXC{$\sigma : \Vars\ \Delta\ \Gamma$}
    \TIC{$\scv\ v^{+\sigma}$}
    \DP
  \]
\end{lemma}
\begin{proof}
  By induction over the type.
  At the base type, stability of $\q$ under weakening is used.
  For function types, the proof is almost immediate, since
  the definition of strong computability already accounts for weakening.
\end{proof}

Using soundness, we can prove that strong computability is a mere proposition.
\begin{lemma}
  \label{lem:scvProp}
  \[
    \AXC{$x,y : \scv\ v$}
    \UIC{$x \equiv y$}
    \DP
  \]
\end{lemma}
\begin{proof}
  By induction over the type.
  At the base type, we use soundness of $\q$, that is
  \[
    \AXC{$\q\ v \Da n$}
    \AXC{$\q\ v \Da m$}
    \BIC{$n \equiv m$}
    \DP
  \]
  which follows easily from lemmas~\ref{lem:quoteCompl} and~\ref{lem:nfembinj}.
  For function types, lemma~\ref{lem:evalSound} is used.
\end{proof}

The most important lemma regarding strong computability is that $\q$
terminates on all strongly computable values. We simultaneously prove
that any neutral value on which $\qn$ terminates is strongly computable.
\begin{lemma}
  \label{lem:scvqu}
  \[
    \AXC{$v : \Val\ \Gamma\ A$}
    \AXC{$\scv\ v$}
    \RightLabel{(quote)}
    \BIC{$\Sigma(n : \Nf\ \Gamma\ A),\ \q\ v \Da n$}
    \DP
  \]
  \[
    \AXC{$v : \NV\ \Gamma\ A$}
    \AXC{$\Sigma(n : \NN\ \Gamma\ A),\ \qn\ v \Da n$}
    \RightLabel{(unquote)}
    \BIC{$\scv\ (\neu\ v)$}
    \DP
  \]
\end{lemma}
\begin{proof}
  By mutual induction on the type $A$. The base cases are trivial by
  definition of strong computability. Consider a function type $A \Ra B$.
  \begin{description}
  \item[(quote)] Let $f$ be a strongly computable value of type $A \Ra B$.
    We prove that $\q$ on $f$, which is (informally) defined as
    \[ \lam (\q\ (f^{+A}\ @\ \neu\ (var\ \vz))) \]
    terminates. In that expression, the variable $\vz$ has type $A$. Furthermore
    $\qn$ trivially terminates on variables, hence $(unquote)$ implies by
    induction hypothesis that $\neu\ (\var\ \vz)$ is strongly computable.  

    Then by definition of strong computability $f^{+A}\ @\ \neu\ (\var\ \vz) \Da v$
    holds for a strongly computable $v$ of type $B$, and $\q\ v \Da n$ follows
    by induction hypothesis $(quote)$. Therefore, $\q\ f \Da (\lam\ n)$.
  \item[(unquote)] Assume $\qn\ f \Da n$ with $f : \NV\ \Gamma\ (A \Ra B)$.
    We prove that $\neu\ f$ is strongly computable.
    Let $\sigma : \Vars\ \Delta\ \Gamma$ and $v : \Val\ \Delta\ A$ strongly
    computable. Then
    \[ (\neu\ f^{+\sigma})\ @\ v \Da (\neu\ (\app\ f^{+\sigma}\ v)) \]
    is immediate since $f$ is neutral.
    Furthermore, by induction hypothesis $(unquote)$ and definition of $\qn$,
    to prove that $\neu\ (\app\ f^{+\sigma}\ v)$ is strongly computable, it suffice
    to check that $\qn$ terminates on $f^{+\sigma}$ and $\q$ terminates on $v$.
    The former holds by hypothesis using that $\qn$ is stable by weakening,
    while the latter holds by induction hypothesis $(quote)$.
  \end{description}
\end{proof}

Strong computability is extended to environments in the obvious way.
\begin{alignat*}{1}
  & \sce : \Env\ \Gamma\ \Delta \to \Set \\ &
  \begin{alignedat}{2}
    & \sce\ \epsilon && = \top \\
    & \sce\ (\rho,v) && = \sce\ \rho\ \land\ \scv\ v
  \end{alignedat}
\end{alignat*}
Proving that strong computability of environment is immediate using lemma~\ref{lem:scvProp}.
\begin{lemma}
  \label{lem:sceProp}
  \[
    \AXC{$x,y : \scv\ v$}
    \UIC{$x \equiv y$}
    \DP
  \]
\end{lemma}

The main theorem is that evaluation of terms terminates in strongly computable
environments, and furthermore gives a strongly computable result.
\begin{theorem}
  \label{thm:eval}
  \[
    \AXC{$u : \Tm\ \Gamma\ A$}
    \AXC{$\rho : \Env\ \Delta\ \Gamma$}
    \AXC{$\sce\ \rho$}
    \TIC{$\Sigma(v : \Val\ \Delta\ A)\ \eval\ u\ \rho \Da v\ \land\ \scv\ v$}
    \DP\qquad
    \AXC{$\sigma : \Tms\ \Gamma\ \Theta$}
    \AXC{$\rho : \Env\ \Delta\ \Gamma$}
    \AXC{$\sce\ \rho$}
    \TIC{$\Sigma(\nu : Env\ \Gamma\ \Theta)\ \evals\ \sigma\ \rho \Da \nu\ \land\ \sce\ \nu$}
    \DP
  \]
\end{theorem}
\begin{proof}
  By induction on terms and substitutions. Since terms and substitutions are
  quotient inductive types, a proof by induction must be compatible with all
  equality constructors.

  In our case, we first remark using lemmas~\ref{lem:evalSound}
  and~\ref{lem:scvProp} that the theorem is a mere proposition. As a
  consequence any proof will automatically respect all equality constructors.
  Thus, we can proceed with the rest of the proof by only considering the
  regular constructor, as if our syntax was unquotiented.

  Most cases are extremely straight forward, the result of evaluation is
  obtained by following the definition of the evaluator and applying the
  induction hypothesis, and strong computability of the result comes directly
  from the hypothesis.
  We deal with the case of abstraction and application, which are the only ones
  to make use of the definition of strong computability.
  \begin{itemize}
  \item For an abstraction $\lam u$ of type $A \Ra B$ evaluated in a strongly
    computable environment $\rho : \Env\ \Gamma\ \Theta$, evaluation is trivial
    since it simply yields the closure $\clos\ u\ \rho$. Let us show that this
    closure is strongly computable.

    Let $\sigma : \Vars\ \Delta\ \Gamma$, and $v : \Val\ \Delta\ A$ strongly
    computable. Then lemma~\ref{lem:scvWk} implies that $\rho^{+\sigma},v$ is a
    strongly computable environment, hence by induction hypothesis there exists
    $w$ strongly computable such that $\eval\ u\ (\rho^{+\sigma},v) \Da w$.
    It follows that $(\clos\ u\ \rho^{+\sigma})\ @\ v \Da w$. Finally,
    $(\clos\ u\ \rho)^{+\sigma} \equiv \clos\ u\ (\rho^{+\sigma})$ holds by definition
    of the weakening of values. Therefore $\clos\ u\ \rho$ is strongly computable.
  \item Consider an application $\app\ f$ with $f : \Tm\ \Gamma\ (A \Ra B)$
    evaluated in a strongly computable environment $(\rho,v) : \Env\ \Delta\ (\Gamma,A)$.
    By induction hypothesis, there exists $g : \Val\ \Delta\ (A \Ra B)$
    strongly computable such that $\eval\ u\ \rho \Da g$. Furthermore, because
    $g$ and $v$ are strongly computable, there exist $w$ strongly computable such that
    $g\ @\ v \Da w$. Then we obtain by the definition of the evaluation relation that
    $\eval\ (\app\ f)\ (\rho,v) \Da w$, proving the result.
  \end{itemize}
\end{proof}

Finally, lemma~\ref{lem:scvqu} implies that all variables are strongly
computable, and so is the identity environment. Therefore evaluation in the
identity environment terminates for any term, and gives a strongly computable
result. By lemma~\ref{lem:scvqu} again, $\q$ terminates on said result.
Therefore, normalisation terminates for every term.


\section{Syntax of Type Theory}

\section{BSN for Type Theory}

\bibliography{types.bib}
\end{document}


%%% Local Variables:
%%% mode: latex
%%% TeX-master: t
%%% End:
